\section{Energy resolution boost}
\label{Result}
Assume the number of expected photons $N$ on a PMT obeys Poisson distribution $\pi(\mu_N)$. Energy $E$ of an event is proportional to $N=k\eta E$, in which $k$ is a factor related to light yield and light transportation. The PDE of PMT is $\eta$ and the single PE charge distribution is described using Gaussian distribution $G(\mu_C,\sigma_C^2)$. The output charge distribution $C$ is a hierarchical model and the expectation and variance are
\begin{align}
    E[C]&=\mu_N\mu_C\\
    Var[C]&=\mu_C^2\mu_N+\mu_N\sigma_C^2
\end{align}

To be convenience, $N$ is estimated as $\hat{N}=\frac{C}{\mu_C}$ and $E$ is estimated as $\hat{E}=\frac{\hat{N}}{k}$. The reconstructed energy resolution is 
\begin{equation}
    \frac{\sqrt{Var[\hat{E}]}}{E[\hat{E}]}=\frac{\sqrt{\mu_c^2\mu_N+\mu_N\sigma_c^2}}{\mu_N\mu_c}=\frac{\sqrt{1+(\frac{\sigma_c}{\mu_c})^2}}{\sqrt{\mu_N}}=\frac{\sqrt{1+(\frac{\sigma_c}{\mu_c})^2}}{\sqrt{k\eta E}}
\end{equation}

The energy resolution is dominated by the term $\eta$ and $\frac{\sigma_c}{\mu_c}$, which is estimated as \emph{sample resolutions} $\nu$. Resolution factor $\nu_{E}=\frac{\sqrt{1+(\frac{\sigma_c}{\mu_c})^2}}{\sqrt{\eta}}$ is proportional to the energy resolution and is calculated for reference PMT and MCP-PMTs as shown in Fig.~\ref{fig:EnergyResolution}. Although there exists a long tail in the charge distribution of MCP-PMT, the boost of relative PDE of MCP-PMT leads to a better energy resolution with about 0.1 boosts. The advanced waveform analysis method which models the charge distribution and counts PEs may eliminate the impact of long tail effect and gives a better energy resolution.
\begin{figure}[!htbp]
    \centering
    \includegraphics[width=\MF\textwidth]{figures/result/resolution.pdf}
    \caption{Contour map of $\nu_{E}$ as a function of the \emph{sample resolution} $\nu$ and the relative PDE. The relative PDE of reference PMT is 1.}
    \label{fig:EnergyResolution}
\end{figure}
