\section{Result}
\label{Result}
10 MCP PMT were tested in testing system. The results are summarized in following sections. The mean and standard deviation of parameters tested several times are calculated and shown in following figures.

\subsection{Gain, single PE resolution, P/V ratio, rise time, fall time and FWHM}
\begin{figure}[!htbp]
    \centering
    \begin{subfigure}[b]{0.47\textwidth}
        % \includegraphics[width=\textwidth,page=1]{figures/method/noisebaseline697_219908_2.pdf}
        \caption{Gain distribution for noise and trigger stage}
        \label{fig:gainCompare}
    \end{subfigure}
    \begin{subfigure}[b]{0.47\textwidth}
        % \includegraphics[width=\textwidth,page=3]{figures/method/noisebaseline697_219908_2.pdf}
        \caption{single PE resolution for noise and trigger stage}
        \label{fig:speresolutionCompare}
    \end{subfigure}
    \begin{subfigure}[b]{0.47\textwidth}
        % \includegraphics[width=\textwidth,page=3]{figures/method/noisebaseline697_219908_2.pdf}
        \caption{P/V ratio}
        \label{fig:PVCompare}
    \end{subfigure}
    \begin{subfigure}[b]{0.47\textwidth}
        % \includegraphics[width=\textwidth,page=3]{figures/method/noisebaseline697_219908_2.pdf}
        \caption{P/V ratio}
        \label{fig:FWHMCompare}
    \end{subfigure}
\end{figure}
Fig~\ref{fig:gainCompare} and Fig~\ref{fig:speresolutionCompare} indicates that the results of noise stage and trigger stage are consistent. Due to the rate of trigger stage is larger than noise stage, the P/V ratio of trigger ratio is better than noise stage.

\subsection{DCR}
\begin{figure}[!htbp]
    \centering
    % \includegraphics[width=\textwidth]{figures/method/triggerwave.pdf}
    \caption{DCR}
    \label{fig:DCRCompare}
\end{figure}
\subsection{TTS}
\begin{figure}[!htbp]
    \centering
    % \includegraphics[width=\textwidth]{figures/method/triggerwave.pdf}
    \caption{TTS}
    \label{fig:TTSCompare}
\end{figure}
\subsection{Paramerters of SER}
\begin{figure}[!htbp]
    \centering
    \begin{subfigure}[b]{0.47\textwidth}
        % \includegraphics[width=\textwidth,page=3]{figures/method/noisebaseline697_219908_2.pdf}
        \caption{$\tau$}
        \label{fig:tauCompare}
    \end{subfigure}
    \begin{subfigure}[b]{0.47\textwidth}
        % \includegraphics[width=\textwidth,page=3]{figures/method/noisebaseline697_219908_2.pdf}
        \caption{$\sigma$}
        \label{fig:sigmaCompare}
    \end{subfigure}
\end{figure}
\subsection{Pre-pulse and after-pulse}
\begin{figure}[!htbp]
    \centering
    \begin{subfigure}[b]{0.47\textwidth}
        % \includegraphics[width=\textwidth,page=3]{figures/method/noisebaseline697_219908_2.pdf}
        \caption{pre-pulse ratio}
        \label{fig:prepulseCompare}
    \end{subfigure}
    \begin{subfigure}[b]{0.47\textwidth}
        % \includegraphics[width=\textwidth,page=3]{figures/method/noisebaseline697_219908_2.pdf}
        \caption{after-pulse ratio}
        \label{fig:afterpulseCompare}
    \end{subfigure}
\end{figure}
\subsection{PDE}
\begin{figure}[!htbp]
    \centering
    % \includegraphics[width=\textwidth]{figures/method/triggerwave.pdf}
    \caption{PDE}
    \label{fig:PDECompare}
\end{figure}
The results shows P/V of MCP PMTs are better than reference PMT.

The results of noise and trigger methods are consist.
\subsection{Energy resolution boost}
Assume the number of expected photons $N$ on PMT is $\mu_N$ and obey poisson distribution $\pi~(\mu_N)$. Energy $E$ of Event proportional to $N=kE$ and $k$ is the factor which is relate to light yield and light transportation. The phton detection efficiency of PMT is $\eta$ and the sigle PE charge distribution is Gaussian distribution $G(\mu_c,\sigma_c)$. The output charge distribution $C$ is hierachical model and the expectation and variancen are
\begin{align}
    E[C]&=\mu_N\mu_c\\
    Var[C]&=\mu_c^2\mu_N+\mu_N\sigma_c^2
\end{align}
To be convenience, $N$ is estimated as $\hat{N}=\frac{C}{\mu_c}$ and $E$ is estimated as $\hat{E}=\frac{\hat{N}}{k}$. For event of energy $E$, the reconstructed energy resolution is 
\begin{equation}
    \frac{\sqrt{Var[\hat{E}]}}{E[\hat{E}]}=\frac{\sqrt{\mu_c^2\mu_N+\mu_N\sigma_c^2}}{\mu_N\mu_c}=\frac{\sqrt{1+(\frac{\sigma_c}{\mu_c})^2}}{\sqrt{\mu_N}}=\frac{\sqrt{1+(\frac{\sigma_c}{\mu_c})^2}}{\sqrt{kE}}
\end{equation}

The energy resolution is dominated by sigle PE charge distribution, which is illustrated in the Fig~\ref{fig:EnergyResolution}. Althogh there exist a long tail in charge distribution of MCP-PMT, $\frac{\sigma_c}{\mu_c}$ of MCP-PMT is better than reference PMT.
\begin{figure}[!htbp]
    \centering
    % \includegraphics[width=\textwidth]{figures/method/triggerwave.pdf}
    \caption{Energy resolution}
    \label{fig:EnergyResolution}
\end{figure}