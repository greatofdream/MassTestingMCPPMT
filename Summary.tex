\section{Summary}
\label{Summary}
\begin{table}
    \centering
    \caption{Summary of important parameters of 9 MCP-PMTs}
    \label{tab:summary}
    \begin{tabular}{|c|c|c|c|c|c|}
        \hline
        $\overline{Q}/Q_0$&$\nu_0$&$\nu$&TTS/ns&DCR/kHz&relative PDE\\
        \hline
        1.8$\pm$0.1&0.25$\pm$0.02&0.69$\pm$0.03&1.74$\pm$0.05&5.3$\pm$1.7&1.71$\pm$0.06\\
        \hline
    \end{tabular}
\end{table}

The characteristics of the nine MCP-PMTs are summarized in Table~\ref{tab:summary}, which can be served as inputs to detector simulation and data analysis. A new calibration method based on regression gives the average relative PDE of the MCP-PMT to be about 1.7 times the reference PMT. The long tail in the charge distribution is countered by the high PDE, resulting in an overall boost in energy resolution. We conclude that the new 8-inch GDB-6082 MCP-PMT from NNVT is suitable for the upcoming JNE.