\section{Summary}
\label{Summary}
\begin{table}
    \centering
    \caption{Summary of important parameters of the MCP-PMTs}
    \label{tab:summary}
    \begin{tabularx}{\textwidth}{c|r @{$\pm$} l cXc}
        \hline\hline
        Parameters&\multicolumn{2}{c}{Value}&Criteria&Notes&Section\\
        \hline
        $\overline{Q}/Q_0$&1.8&0.1&&Entire-Sample to Main-Peak Gain Ratio&\ref{sec:noisepeak}\\
        $\nu_0=\sigma_{Q_0}/Q_0$&0.25&0.02&&Peak Resolution&\ref{sec:noisepeak}\\
        $\nu=\left.\sqrt{s^2_{Q}}\middle/\overline{Q}\right.$&0.69&0.03&&Sample Resolution&\ref{sec:noisepeak}\\
        $N^{\mathrm{1e}}/N^{\mathrm{hit}}$&0.59&0.02&&Main-peak Fraction&\ref{sec:noisepeak}\\
        P/V&5.9&1.4&$>5$&Peak-to-Valley Ratio&\ref{sec:PV}\\
        $t_r$/ns&3.71&0.15&$<4$&Rise Time&\ref{sec:SER}\\
        $t_f$/ns&15.6&1.8&$<20$&Fall Time&\ref{sec:SER}\\
        \hline
        $\sigma_{\mathrm{SER}}$/ns&1.63&0.06&& \multirow{2}{=}{Shape Parameters of SER} & \ref{sec:SER}\\
        $\tau_{\mathrm{SER}}$/ns&7.2&1.1&&&\\
        \hline
        TTS/ns&1.73&0.08&$<1.8$&Transit Time Spread &\ref{sec:TTS}\\
        DCR/kHz&5.8&1.6&$\sim 5$&Dark Count Rate&\ref{sec:dcr}\\
        $P_{\mathrm{pre}}$&1E-6&6E-6&$<0.001$&Pre-Pulse Probability&\ref{sec:afterpulse}\\
        $P_{\mathrm{after}}$&0.009&0.005&$<0.048$&After-Pulse Probability&\ref{sec:afterpulse}\\
        $\epsilon^0$&1.71&0.06&$>1.6$&Relative PDE&\ref{sec:PDE}\\
        \hline\hline
    \end{tabularx}
\end{table}

The characteristics of the nine MCP-PMTs discussed are summarized in Table~\ref{tab:summary}, which can be served as inputs to detector simulation and data analysis. A new calibration method based on regression in this study gives the average relative PDE of the MCP-PMT to be about 1.7 times the reference PMT. The long tail in the charge distribution is countered by the high PDE, resulting in an overall boost in energy resolution. We conclude that the new 8-inch GDB-6082 MCP-PMT from NNVT is suitable for the upcoming JNE.