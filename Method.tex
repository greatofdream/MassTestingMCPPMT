\section{Testing and characterization method}
\label{Method}
The setup and analysis method of the noise stage is discussed in sec.~\ref{sec:noisestage}. There exist some differences between the noise stage and the laser stage. The analysis method of the laser stage is concentrated on the differences in sec.~\ref{sec:laserstage}.
\subsection{Noise stage}
\label{sec:noisestage}
The sample window size $T_{\mathrm{wave}}$ is set as \SI{600}{ns}. For \SI{20}{kHz} dark noise rate, the expected dark noise pulse number is 0.012 for each wave, which means the probability of 2 or more pulses is about 0.012 times of probability of 1 pulse. The maximum peak in each waveform is extracted as the noise pulse candidates.
\subsubsection{Baseline}
Due to the offset mentioned in sec.~\ref{sec:setup}, the ADC value of baseline is not zero. A procedure to determine the baseline is developed, which comprised three steps and is shown in Fig.~\ref{fig:baseline1}
\begin{figure*}[!htbp]
    \centering
    \includegraphics[width=0.8\textwidth,page=1]{figures/method/noisebaseline697_219908_2.pdf}
    \caption{An example of a large waveform in noise stage. The vertical green line indicates the peak of pulse. The horizonal blue dash line is the baseline. The rise time and fall time are the width of pink and green rectangle.}
    \label{fig:baseline1}
\end{figure*}

1. An interval of the time window $[-t_s,-10]$\,ns ($t_s\in[110,200]$) relative to pulse peak $t_p$ is selected for calculation of baseline. If $t_s < 110$ due to the peak is close to the start of waveform, another interval $[t_p+100,t_p+200]$\,ns is append to the total interval. Average $\mu_{\mathrm{b0}}$ and standard deviation $\sigma_{\mathrm{b0}}$ of amplitudes in the interval are calculated.

2. An baseline threshold filter $[\mu_{\mathrm{b0}}-\max(\min(5\sigma_{\mathrm{b0}},3),1)]$ is used to remove potential signal and reserve the baseline when $\sigma_{\mathrm{b0}}$ is small.

3. The rest amplitudes are fitted with a gaussian function G$(\mu_{\mathrm{bf}},\sigma_{\mathrm{bf}})$ using unbinned likelihood. $\mu_{\mathrm{bf}}$ and $\sigma_{\mathrm{bf}}$ are accurate at most time. However, when there exists a large wave in the time interval selected in step 1, a bias will be introduced for $\sigma_{\mathrm{bf}}$.

4. Another amplitude filter $[\mu_{\mathrm{bf}}-\min(5\sigma_{\mathrm{bf}},3)]$ is used to reselect the signal area and those areas are padding \SI{10}{ns} at both ends to remove rising edge and falling edge. The rest wave is used to estimate baseline $\mu_b$ and the standard deviation of baseline $\sigma_b$.

\subsubsection{Peak and charge spectrum}
\label{sec:noisepeak}

The peak $V_p$ of a pulse is the difference between the baseline and minimum of pulse. The charge of pulses is calculated using integration in a time window $[-15, 75]$\,ns relative to the peak location $t_p$ of the signal as shown in Fig.~\ref{fig:baseline1} considering the rise time and fall time distribution. The equivalent charge $C_{\mathrm{equ}}$ is defined as the summation of the amplitudes in the integration interval. Considering the input impedance is \SI{50}{\Omega}, the true charge follows Equ.~\eqref{equ:charge} 
\begin{equation}
    \label{equ:charge}
    C = \frac{C_{\mathrm{equ}}}{50 \Omega}
\end{equation}

Fig.~\ref{fig:charge} shows the distribution of equaivalent charge. The pedestal is a set of waveform with no signal, the expected charge of which is zero. To remove the influence of pedestal, peaks large than \SI{3}{ADC} and charge large than 0.25\,p.e. is selected. Mean $\mu_{C_{\mathrm{t}}}$ and sample variance $s^2_{C_{\mathrm{t}}}$ of charge of selected pulses are calculated to represent the characteristics of total charge distribution of MCP-PMT. Due to the influence of long tail, the mean charge of single PE is larger than the postion of the first main peak of charge distribution in Fig.~\ref{fig:charge}. The peak amplitude distribution is shown in Fig.~\ref{fig:peak} and the bin width of histogram is 1ADC.
\begin{figure}[!htbp]
    \centering
    \begin{subfigure}[t]{0.49\textwidth}
        \includegraphics[width=\textwidth]{figures/method/charge697.pdf}
        \caption{}%PM
        \label{fig:charge}
    \end{subfigure}
    \begin{subfigure}[t]{0.49\textwidth}
        \includegraphics[width=\textwidth]{figures/method/peak697.pdf}
        \caption{}%PM
        \label{fig:peak}
    \end{subfigure}
    \caption{(a) Charge distribution of an example PMT. The vertical green dash line is threshold cut for charge. Origin histogram is the selected entries with peak selection. The red line is the fit Gaussian function for peak. The green line is the fit parabolic function for vally. (b) Peak distribution of an example PMT. The vertical green dash line is threshold cut for peak. Origin histogram is the selected entries with charge selection.}
\end{figure}
\subsubsection{Gain and single PE resolution}
\label{sec:noisegain}
There exist a long tail in charge distribution as shown in the histogram with \SI{1}{ADCns} in Fig.~\ref{fig:charge}. To describe the energy resolution of the first peak of the charge distribution, a fit interval $[-35, 35]$\,ADCns relative to the largest count bin of histogram is used. A gaussian function G$(\mu_{C_1},\sigma^2_{C_1})$ is used to fit the binned data via the modified least-square method \cite{Cowan1998StatisticalDA} to capture the main peak of charge distribution of single PE. The gain of main peak $G_1$ and gain of single PE $G$ are calculated as the following equation
\begin{align}
    G_1&=\frac{\mu_{C_1}}{e\times 50\Omega} \\
    G &= \frac{\mu_{C_t}}{e\times 50\Omega}
\end{align}
in which $e$ is the charge of an electron.

The main peak resolution and the single PE resolution are defined as
\begin{align}
    \mathrm{Res}_1&=\frac{\sigma_{C_1}}{\mu_{C_1}}\\
    \mathrm{Res}&=\frac{\sqrt{s^2_{C_t}}}{\mu_{C_t}}
\end{align}

\subsubsection{Peak-to-valley (P/V) ratio}
A parabolic function is fitted to the vally interval $[-15, 30]$\,ADCns relative to the least count bin of histogram between pedestal and SPE peak as shown in Fig.~\ref{fig:charge}. The local minimum $N_v$ of charge spectrum is calculated as the minima of parabolic function. The $N_p$ is the peak of gaussian function described in sec.~\ref{sec:noisegain}. The peak-to-valley ratio is equal to  
\begin{equation}
    \mathrm{P/V}=\frac{N_p}{N_v}
\end{equation}
The P/V show the ability of discrimination between electronic noise and true signal.
\subsubsection{rise time, fall time and full width at half maximum (FWHM)}
\begin{figure}[!htbp]
    \centering
    \begin{subfigure}[t]{0.49\textwidth}
        \includegraphics[width=\textwidth]{figures/method/FWHM697.pdf}
        \caption{}
        \label{fig:risefallFWHM}
    \end{subfigure}
    \begin{subfigure}[t]{0.49\textwidth}
        \includegraphics[width=\textwidth]{figures/method/triggerFWHM.pdf}
        \caption{}
        \label{fig:triggerFWHM}
    \end{subfigure}
    \caption{(a) An example of time characteristics in noise stage. (b)An example of time characteristics in trigger stage.}
\end{figure}
As shown in Fig.~\ref{fig:baseline1}, $t^{10}_r$, $t^{50}_r$, $t^{90}_r$ are the time of 10\%, 50\% and 90\% amplitude in the rising edge and $t^{10}_f$, $t^{50}_f$, $t^{90}_f$ are the time of 10\%, 50\% and 90\% amplitude in the falling edge which are acquired via interpolation method. The rise time, fall time, and FWHM are calculated as following:
\begin{align}
    t_r &= t^{90}_r - t^{10}_r\\
    t_f &= t^{10}_f - t^{90}_f\\
    \mathrm{FWHM} &= t^{50}_f - t^{50}_r
\end{align}

Due to some pulse are close to the edge of waveform, the rising or falling edge of those pulses are cut by the time window. Therefore only pulses of which peak positions in $[15, T_{\mathrm{wave}}-75]$\,ns are selected. Fig.~\ref{fig:risefallFWHM} shows the distribution of rise time, fall time and FWHM of an example PMT. The mean and standard deviation are calculated in condtion of a amplitude threshold $V_{t}>3\mathrm{ADC}$ and charge threshold 0.25\,p.e..

\subsubsection{Dark count rate (DCR)}
An amplitude threshold $V_{p}>3\mathrm{ADC}$ and charge threshold 0.25\,p.e. as shown in Fig.~\ref{fig:peak} are selected to discriminate the dark noise with fluctuation of baseline. The DCR equals to the following equation
\begin{equation}
    \mathrm{DCR/kHz} = \frac{N_{\mathrm{noise}}}{N_{t}}\frac{1}{T_{\mathrm{wave}}/\mathrm{ns}}\times 10^{6}
\end{equation}
in which $N_{\mathrm{noise}}$ is the noise number and $N_{t}$ is the total number of waveforms.

\subsection{Laser stage}
\label{sec:laserstage}

\begin{figure*}[!htbp]
    \centering
    \includegraphics[width=0.8\textwidth]{figures/method/triggerwave.pdf}
    \caption{An example of waveform in laser stage. The orange line is the trigger waveform. Two blue horizonal dash line are the upper and lower values of step wave. Green cross point is the interpolation for trigger time. The magnified axes shows the PMT waveform with a signal. Red and green vertical dash line are the time of 10\% of rising edge and pulse peak.}
    \label{fig:triggertime}
\end{figure*}

To yield SPE events, the laser intensity was adjusted to a level only about one out of 20 trigger signals lead to a signal. The window size $T_{\mathrm{wave}}$ is set to \SI{10400}{ns} and the rising edge of trigger waveform is at about \SI{200}{ns}. The trigger from the laser system is a step wave. The vertical center of rising edge is interpolated to get the trigger time $t_{\mathrm{trig}}$ as shown in Fig.~\ref{fig:triggertime}.

The triggered pulse is mainly centered in the time interval between $[t_{\mathrm{trig}}, 600]$\,ns dependent on the length of cable. The maximum peak is found in the window of $[t_{\mathrm{trig}}, 600]$\,ns to roughly extract the peak position. A gaussian function G$(\mu_t^0,\sigma_t^0)$ is unbinned fit to the distribution of peak location of pulses whose peak large than \SI{5}{ADC} for each PMT as shown in Fig.~\ref{fig:peaklocation}.

A time interval $[\mu_{t0}-3\sigma_{t0}, \mu_{t0}+3\sigma_{t0}]$ is used for selecting a waveform dataset for each PMT, in which peaks of the triggered wave candidates fall. All the characterizations are recalculated again with the new time cut which reduces the impact of dark noise.

\begin{figure}[!htbp]
    \centering
    \includegraphics[width=0.5\textwidth]{figures/method/triggerpeakpos.pdf}
    \caption{Peak location distribution of an example PMT. A gaussian function is fit to the distribution to acquire an optimized time cut.}%PM
    \label{fig:peaklocation}
\end{figure}

\subsubsection{Peak and charge spectrum, Gain and single PE resolution, Peak-to-valley (P/V) ratio}
\label{sec:triggerpeak}
The peak, charge, gain, single PE resolution and P/V ratio calculation method are same as above in sec.~\ref{sec:noisepeak}. The charge distribution with amplitude cut is shown in Fig.~\ref{fig:triggercharge}. The peak amplitude distribution is shown in Fig.~\ref{fig:triggerpeak}. Due to the statistics is far larger than noise mode and the ratio of dark noise signal is smaller, the P/V ratio is better than noise mode.
\begin{figure}[!htbp]
    \centering
    \begin{subfigure}[b]{0.49\textwidth}
        \includegraphics[width=\textwidth]{figures/method/triggercharge.pdf}
        \caption{}%PM
        \label{fig:triggercharge}
    \end{subfigure}
    \begin{subfigure}[b]{0.49\textwidth}
        \includegraphics[width=\textwidth]{figures/method/triggerpeak.pdf}
        \caption{}%PM
        \label{fig:triggerpeak}
    \end{subfigure}
    \caption{(a) Charge distribution of an example PMT in trigger stage. (b) Peak distribution of an example PMT in trigger stage. The selection and fit method is consist with noise stage.}
\end{figure}

\subsubsection{Rise time, fall time and full width at half maximum (FWHM)}
\label{sec:triggerFWHM}
The definitions of rise time, fall time and FWHM are consist with noise mode. Fig.~\ref{fig:triggerFWHM} shows the distribution of a PMT. A amplitude threshold $V_{p}>3\mathrm{ADC}$ and charge threshold 0.25\,p.e. is used to decide whether a pulse is generated in the target time interval.

\subsubsection{Transit time spread (TTS)}
The TTS is the spread of photo-electron transit time (TT), which represents resolution of timing. The transit time cannot be measured directly, while the trigger time of laser and time of pulse can be measured. A relative transit time ($\mathrm{TT}_r$) is definited as the time between trigger time $t_{\mathrm{trig}}$ and $t_r^{10}$. The $\mathrm{TT}_r$ distribution using above criteria in sec.~\ref{sec:triggerFWHM} is histogramed with \SI{0.2}{ns} bin width. A gaussian fuction G$(\mu_{\mathrm{TT}},\sigma_{\mathrm{TT}}^2)$ is binned fitted to the histogram with interval $[-2,+2]$\, ns relative to the maximum bin as shown in Fig.~\ref{fig:triggerTTS}. Due to the time precision of FADC is \SI{1}{ns}, the TT distribution contains two peak when bin width is smaller than \SI{1}{ns}. TTS is defined as
\begin{equation}
    \mathrm{TTS}=2\sqrt{2\ln(2)}\sigma_{\mathrm{TT}}
\end{equation}
Fig.~\ref{fig:triggerTTS2d} indicates a long tail in charge distribution.
\begin{figure}[!htbp]
    \centering
    \begin{subfigure}[t]{0.49\textwidth}
        \includegraphics[width=\textwidth]{figures/method/triggerTTS.pdf}
        \caption{}
        \label{fig:triggerTTS}
    \end{subfigure}
    \begin{subfigure}[t]{0.49\textwidth}
        \includegraphics[width=\textwidth]{figures/method/triggerTTS2d.pdf}
        \caption{}
        \label{fig:triggerTTS2d}
    \end{subfigure}
    \caption{(a) TT distribution of an example PMT in trigger stage. (b) 2d histogram of TTS and charge in trigger stage}
\end{figure}

\subsubsection{Single electron response (SER)}
The pulses are selected by dedicated cuts:
\begin{itemize}
    \item[1] The amplitude and charge of pulse candidate has to be fulfill the criteria in sec.~\ref{sec:triggerFWHM}.
    \item[2] The FWHM of candidate pulse should exceed \SI{5}{ns} to avoid noise.
    \item[3] To focus on the pulse in the main peak of charge distribution, an upper charge threshold should set. Besides, pulse with small charge is influenced by noise. A charge filter $[0.5C_1, 1000]$ is used for reliable results.
\end{itemize}

\begin{figure}
    \centering
    \begin{subfigure}[t]{0.47\textwidth}
        \includegraphics[width=\textwidth]{figures/method/triggerSER.pdf}
        \caption{}
        \label{fig:triggerser}
    \end{subfigure}
    \begin{subfigure}[t]{0.47\textwidth}
        \includegraphics[width=\textwidth]{figures/method/triggerSER2d.pdf}
        \caption{}
        \label{fig:triggerser2d}
    \end{subfigure}
    \caption{(a) An example pulse and fit result in trigger stage. (b) Paratemter distribution for an example PMT.}
\end{figure}

A gaussian convoluted with an exponential function Equ~\eqref{equ:ser} is used to fit the SER which is shown in Fig.~\ref{fig:triggerser}
\begin{equation}
    \label{equ:ser}
    Gaus(0,\sigma_t^2)\otimes\theta(t-\mu_t)\frac{1}{\tau}e^{-\frac{t-\mu_t}{\tau}}
\end{equation}
The average of $\sigma$ and $\tau$ canbe used to simulate the SER. 
\subsubsection{Pre-pulse and after-pulse}
The generation of pre-pulses is due to photons hit on the MCP or the first dynode directly rather than the photocathode \cite{JUNOMassTesting}. The amplitude of pre-pulses are smaller than normal signal and appear before the main pulse. This ratio is related to the intensity of light source.
\begin{figure*}
    \centering
    \includegraphics[width=0.8\textwidth]{figures/method/triggerAfterpulseSchema.pdf}
    \caption{An example waveform. Green and red vertical dash line is the rising edge of trigger waveform and 10\% of rising edge of main pulse.}
    \label{fig:afterpulseSchema}
\end{figure*}

Afterpulse are generated due to the ionization of gaseous impurities between the cathode and first dynode when photo-electrons go through \cite{Coates_1973}. These ions hit back on the photocathode and generate electrons. \ce{H^+}, \ce{He^+}, \ce{O^+} are the major ions contributing to afterpulse and the relation between time and ions (\ce{^Z_MX}) is $\sqrt{\frac{M}{Z}}$\cite{Coates_1973}. Due to these ions are heavier than electron, the travel time is in the scale of \si{us}\footnote{The velocity of ions is about \SI{1000}{km/s} and size of PMT is about \SI{0.1}{m}, thus the transit time is about \SI{0.1}{us}}. The after pulse is searched in a window \SIrange{250}{10000}{ns} after the main pulse as shown in Fig.~\ref{fig:afterpulseSchema}. The $t$ and $Q$ are calculated as the max peak and integrated in the $[-15,75]$ ns window relative to peak position as shown violet area in Fig.~\ref{fig:afterpulseSchema}
\begin{figure*}[!htbp]
    \centering
    \includegraphics[width=0.8\textwidth]{figures/method/triggerAfterpulse1d.pdf}
    \caption{Time distribution of after pulse. The orange and blue histogram are pre-pulse and after-pulse. The green line is the fit result for after pulse. The blank area around the \SI{0}{ns} is the main pulse which is not shown in this figure.}
    \label{fig:afterpulse1d}
\end{figure*}
% waveform analysis

Afterpulse is categorized into several kinds. Fig.~\ref{fig:afterpulse1d} indicates several typical after-pulse peaks in time around \SI{300}{ns}, \SI{500}{ns}, \SI{1200}{ns} and \SI{1700}{ns}. Considering the different mass of ions, these peaks may originates from \ce{H^+}, \ce{He^{2+}}, \ce{O^+}, and \ce{O_2^+} or unknown ions.
The ratio of prepulse $R_{pre}$ is calculated in time interval [-250,-50]\,ns and after pulse $R_{after}$ is calculated in time interval [-300,-10000]\,ns. 

The time distribution and ratio of afterpulse are parameterized using 4 gaussian function to model the four peak as show in Fig.~\ref{fig:afterpulse1d} and the fit equation is as following
\begin{equation}
    N_{\mathrm{trig}}(A_1G(t_1,\sigma_1^2)+A_2G(t_2,\sigma_2^2)+A_3G(t_3,\sigma_3^2)+A_4G(t_4,\sigma_4^2))
\end{equation}
in which $A_i$, $t_i$, and $\sigma_i$ are the ratio, time, and width of each peak of after pulse.
\begin{figure*}[!htbp]
    \centering
    \includegraphics[width=0.8\textwidth]{figures/method/triggerAfterpulse2d.pdf}
    \caption{time and charge distribution of after pulse}
    \label{fig:afterpulse2d}
\end{figure*}
Fig.\ref*{fig:afterpulse2d} indicates that the after pulse contains some very large signal.
\subsubsection{relative photon detection efficiency}
The DCR is omit in PDE calculation. To measure PDE, the intensity of light and light allocation ratio of each channel need to be calibrated. For example JUNO fixed one reference PMT and other reference PMTs are circulated through all channels \cite{Wonsak_2021}. A new method is designed to reduce the number of reference PMT and combine all test runs to do calibration.

Note $n,j,k$ ($n=0,...,N-1, j=0,...,J-1, k=0,...,K-1$) is the indicator of test run, channel of splitter and PMT. Intensity of light is $I_n$, light allocation ratio is $\alpha_j$ and photon detection efficiency is $\eta_k$. Assume $\alpha_j$s are stable among different test runs. $N_t$ is the total number of waveforms. The photon numbers in each waveform obey Poisson distribution $\pi(p_{njk})$, in which $p_{njk}=I_n\alpha_j\eta_k$. The trigger rate of nth run, kth PMT in jth channel is
\begin{equation}
    \label{equ:pderate}
    R_{njk}=1-e^{-p_{njk}}
\end{equation}

The trigger number of kth PMT in nth run with jth splitter obey Binomial distribution $B(R_{njk},N_t)$. The observed variable is trigger number $N_{\mathrm{trig}}$ and total number $N_{t}$ of waveform. For convenience, 0th PMT is the reference PMT. Note $\alpha_j^0=\frac{\alpha_j}{\alpha_0}$, $\eta_k^0=\frac{\eta_k}{\eta_0}$, $I_n^0=I_n\alpha_0\eta_0$, $i\equiv njk$. $p_{njk}$ can be transfer to Equ~\eqref{equ:pdelograte}
\begin{equation}
    \label{equ:pdelograte}
    \mathrm{log}(p_{njk})=\mathrm{log}(I_0\alpha_0\eta_0)+\mathrm{log}(I_n^0)+\mathrm{log}(\alpha_j^0)+\mathrm{log}(\eta_k^0)
\end{equation}
The relationship between $R_{njk}$ and parameters is
\begin{equation}
    \label{equ:linkfunction}
    R_{njk}=1-e^{-e^{\mathrm{log}(I_0\alpha_0\eta_0)+\mathrm{log}(I_n^0)+\mathrm{log}(\alpha_j^0)+\mathrm{log}(\eta_k^0)}}
\end{equation}
To fit the parameters, a likelihood is constructed as following
\begin{equation}
    \label{equ:likelihood}
    \mathcal{L}=\prod_{njk}{R_{njk}^{N_{\mathrm{trig}_{njk}}}(1-R_{njk})^{N_{t_{njk}}-N_{\mathrm{trig}_{njk}}}}
\end{equation}

The relationship in Equ.~\eqref{equ:linkfunction} matchs GLM of Binomial exponential family distribution with Cloglog link function\cite{glm}. The GLM is used to maximize the likelihood in Equ.~\eqref{equ:likelihood} and calibrate splitter ratio and relative photon detection efficiency.