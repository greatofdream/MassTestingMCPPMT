\section{Testing and characterization method}
\label{Method}

\subsection{Noise stage}
The window size is \SI{600}{ns}. For 20kHz dark noise rate, the expected dark noise pulse number is 0.012, which means the probability of 2 or more pulse is about 0.012 of probability of 1 pulse. The maximum peak in each waveform are extracted as the noise pulse candidates.
\subsubsection{Baseline}
Due to the offset mentioned in sec~\ref{sec:setup}, the ADC value of baseline is about 950. To get baseline, the pulse area should be removed. A procedure to determine the baseline is developed, which comprised three steps and shown in Fig~\ref{fig:baseline1}
\begin{figure}[!htbp]
    \centering
    \begin{subfigure}[b]{\textwidth}
        \includegraphics[width=\textwidth,page=1]{figures/method/noisebaseline697_219908_2.pdf}
        \caption{An example of waveform in noise stage}
        \label{fig:baseline1}
    \end{subfigure}
    \begin{subfigure}[b]{\textwidth}
        \includegraphics[width=\textwidth,page=3]{figures/method/noisebaseline697_219908_2.pdf}
        \caption{An example of waveform without baseline in noise stage}
        \label{fig:baseline2}
    \end{subfigure}
\end{figure}

1. An interval in the time window $[-t_s,-10]$ ($t_s\in[-200,-110]$) ns relative to pulse peak $t_p$ is selected. If $t_s<110$ when the peak is close to the start of waveform, another interval $[100,200]$ ns relative to $t_p$ is appended to expand the total interval. Average $\mu_b^0$ and standard deviation $\sigma_b^0$ of amplitudes are calculated.

2. An amplitude filter $[\mu_b^0-\max(\min(5\sigma_b^0,3),1)]$ is used to remove most of signal and reserve the baseline when $\sigma_b^0$ is small.

3. The rest amplitudes are fitted with a gaussian function G$(\mu_b^f,\sigma_b^f)$ using unbinned likelihood. $\mu_b^f$ and $\sigma_b^f$ are accurate at most time. When there exsit a large wave in the time interval selected in step1, a bias will be introduced for $\sigma_b^f$ which can be seen in Fig~\ref{fig:baselineBias2}.

4. Another amplitude filter $[\mu_b^f-\min(5\sigma_b^f,3)]$ is used to reselect the signal area and those areas are padding \SI{10}{ns} to remove rising edge and falling edge. The rest wave are used to estimate baseline $\mu_b$ and standard deviation of baseline $\sigma_b$.
\begin{figure}[!htbp]
    \centering
    \begin{subfigure}[b]{0.7\textwidth}
        % \includegraphics[width=0.8\textwidth]{figure/facility/facility.pdf}
        \caption{Peak distribution of an example PMT}%PM
        \label{fig:baselineBias1}
    \end{subfigure}
    \begin{subfigure}[b]{0.3\textwidth}
        % \includegraphics[width=0.8\textwidth]{figure/facility/facility.pdf}
        \caption{An example of waveform in noise stage}
        \label{fig:baselineBias2}
    \end{subfigure}
\end{figure}

\subsubsection{Peak and charge spectrum}
\label{sec:noisepeak}
The charge of pulses is calculated using integration in a time window $[-15, 75]$ ns relative to the peak location $t_p$ of the signal (Fig~\ref{fig:baseline2}) considering the rise time and fall time distribution. The equivalent charge $C_{\mathrm{equ}}$ is defined as the summation of the amplitude in the integration interval. Considering the input impedance is \SI{50}{\Omega}, the true charge is calculated as Equ~\eqref{equ:charge} 
\begin{equation}
    \label{equ:charge}
    C = \frac{C_{equ}}{50\Omega}
\end{equation}
Due to the influence of long tail, the mean charge of the output of MCP-PMT cannot use the mean charge of the first peak. Pulses large than \SI{3}{ADC} and charge large than \SI{0.25}{p.e.} is selected and mean $\mu_{select}$ and standard deviation $\sigma_{select}$ of charge are also calculated to represent the characteristics of MCP-PMT.

The peak amplitude distribution is shown in Fig~\ref{fig:peak} and the binwidth of histogram is 1ADC.
\begin{figure}[!htbp]
    \centering
    \begin{subfigure}[t]{0.45\textwidth}
        \includegraphics[width=\textwidth]{figures/method/charge697.pdf}
        \caption{Charge distribution of an example PMT}%PM
        \label{fig:charge}
    \end{subfigure}
    \begin{subfigure}[t]{0.45\textwidth}
        \includegraphics[width=\textwidth]{figures/method/peak697.pdf}
        \caption{Peak distribution of an example PMT}%PM
        \label{fig:peak}
    \end{subfigure}
    \caption{Peak and charge distribution of an example PMT}
\end{figure}
\subsubsection{Gain and single PE resolution}
\label{sec:noisegain}
There exist a long tail in charge distribution as shown in the histogram with \SI{1}{ADCns} in Fig~\ref{fig:charge}. To decrease the influence on the energy resolution, a fit interval $-30, 30$ ADCns relative to the largest count bin of histogram is used. A gaussian function G$(C_1,\sigma_{C_1})$ is used to fit the binned data via modified least-square method to capture the peak of single PE. The gain $G$ is calculated as following equation
\begin{equation}
    G=\frac{C_1}{e}
\end{equation}
in which e is the charge of an electron.

The single PE resolution is defined as
\begin{equation}
    G=\frac{\sigma_{C_1}}{C_1}
\end{equation}
\subsubsection{Peak-to-valley(P/V) ratio}
A parabolic function is fitted to the vally interval ($-15, 30$ ADCns relative to the least count bin of histogram) between pedestal and SPE peak as shown in Fig~\ref{fig:charge}. The local minimum $N_v$ of charge spectrum is calculated as the parabolic function. The $N_p$ is the peak of gaussian function described in sec~\ref{sec:noisegain}. The peak-to-valley ratio is equal to  
\begin{equation}
    P/V=\frac{N_p}{N_v}
\end{equation}
The P/V show the ability of discrimination between electronic noise and true signal.
\subsubsection{rise time, fall time and full width at half maximum(FWHM)}
\begin{figure}[!htbp]
    \includegraphics[width=0.8\textwidth]{figures/method/FWHM697.pdf}
    \caption{An example of FWHM in noise stage}
    \label{fig:risefallFWHM}
\end{figure}
The definitions of rise time, fall time and FWHM are shown in Fig~\ref{fig:baseline1} ($t^r_{10}, t^r_{50}, t^r_{90}$ are the time of 10\%, 50\% and 90\% amplitude). Due to some pulse are close to the edge of waveform, the rising or falling edge are cut by the window. Therefore only peak positions of those pulses in $[15, L_w-75]$ ns window are selected. Fig~\ref{fig:risefallFWHM} shows the rise time, fall time and FWHM distribution of an example PMT.

\subsubsection{Dark count rate}
A amplitude threshold $V_{t}$ and charge threshold \SI{0.25}{p.e.} as shown in Fig~\ref{fig:peak} are selected as the vally of histogram of peak distributuion to discriminate the dark noise and fluctuation of baseline. The DCR equals to $\frac{N_{\mathrm{noise}}}{N_{t}}$, in which $N_{\mathrm{noise}}$ is the noise number and $N_{t}$ is the total number of waveforms.

\subsection{Laser stage}
In order to yield SPE events as signal the laser intensity was adjusted to a level where only
about one out of 20 trigger signals led to a PMT signal.
\begin{figure}[!htbp]
    \centering
    \includegraphics[width=\textwidth]{figures/method/triggerwave.pdf}
    \caption{An example of waveform in laser stage}
    \label{fig:triggertime}
\end{figure}
The window size is \SI{10400}{ns} and the trigger time is at ~\SI{200}{ns}. The trigger is an \SI{1}{V} step wave, the vertical center is interpolated to get the trigger time as shown in Fig~\ref{fig:triggertime}. The trigger pulse are mainly centering in the time interval between [250, 600] ns dependent on the length of cable. The maximum peak are found in the window of [0, 600] ns and extract the peak position. A gaussian function G$(\mu_t^0,\sigma_t^0)$ is unbinned fit to the distribution of peak location of pulses whose peak large than \SI{5}{ADC} for each PMT as shown in Fig~\ref{fig:peaklocation}. A time interval $[\mu_t^0-3\sigma_t^0, \mu_t^0+3\sigma_t^0]$ is used for the waveforms dataset of each PMTs, in which peaks of the trigger wave candidates fall. All the characterizations are recalculated with the new time cut which reduce impact of dark noise as illustracted in \ref{fig:peakselected}.
\begin{figure}[!htbp]
    \centering
    \begin{subfigure}[t]{0.45\textwidth}
        \includegraphics[width=\textwidth]{figures/method/triggerpeakpos.pdf}
        \caption{Peak location distribution of an example PMT}%PM
        \label{fig:peaklocation}
    \end{subfigure}
    \begin{subfigure}[t]{0.45\textwidth}
        % \includegraphics[width=\textwidth]{figures/method/peak697.pdf}
        \caption{Peak distribution of an example PMT}%PM
        \label{fig:peakselected}
    \end{subfigure}
    \caption{Peak and charge distribution of an example PMT}
\end{figure}

\subsubsection{Peak and charge spectrum, Gain and single PE resolution, Peak-to-valley(P/V) ratio}
\label{sec:triggerpeak}
The minimum of the trigger area is selected as pulse candidate. The peak, charge, gain, single PE resolution and P/V ratio calculation method are same as above in sec~\ref{sec:noisepeak}. Due to the statistics is far larger than noise mode and the ratio of none signal is smaller, the P/V ratio is better than noise mode.

The peak amplitude distribution is shown in Fig~\ref{fig:triggerpeak}. The charge distribution with amplitude cut is shown in Fig~\ref{fig:triggercharge}.
\begin{figure}[!htbp]
    \centering
    \begin{subfigure}[b]{0.45\textwidth}
        \includegraphics[width=\textwidth]{figures/method/triggercharge.pdf}
        \caption{Charge distribution of an example PMT}%PM
        \label{fig:triggerpeak}
    \end{subfigure}
    \begin{subfigure}[b]{0.45\textwidth}
        \includegraphics[width=\textwidth]{figures/method/triggerpeak.pdf}
        \caption{Peak distribution of an example PMT}%PM
        \label{fig:triggercharge}
    \end{subfigure}
    \caption{Peak and charge distribution of an example PMT}
\end{figure}

\subsubsection{rise time, fall time and full width at half maximum(FWHM)}
The definitions of rise time, fall time and FWHM are consist with noise mode. Fig~\ref{fig:triggerFWHM} shows the distribution of a PMT.
\begin{figure}[!htbp]
    \includegraphics[width=0.8\textwidth]{figures/method/triggerFWHM.pdf}
    \caption{An example of FWHM in trigger stage}
    \label{fig:triggerFWHM}
\end{figure}
\subsubsection{Transit time spread (TTS)}
The transit time spread (TTS) is the spread of photo-electron transit time (TT), which represents resolution of timing. The transit time cannot be measured directly, while the trigger time of laser and time of pulse can be measured. A relative transit time ($\mathrm{TT}_r$) is definited as the time between trigger time $t_{\mathrm{trig}}$ and $t_{10}^r$. A gaussian fuction G$(\mu_t,\mathrm{TTS}^2)$ is unbinned fitted to the distribution of $\mathrm{TT}_r$.
\begin{figure}[!htbp]
    \centering
    \begin{subfigure}[t]{0.47\textwidth}
        \includegraphics[width=\textwidth]{figures/method/triggerTTS.pdf}
        \caption{An example of TTS in trigger stage}
        \label{fig:triggerTTS}
    \end{subfigure}
    \begin{subfigure}[t]{0.47\textwidth}
        \includegraphics[width=\textwidth]{figures/method/triggerTTS2d.pdf}
        \caption{2d histogram of TTS and charge in trigger stage}
        \label{fig:triggerTTS2d}
    \end{subfigure}
\end{figure}
\subsubsection{Single electron response (SER)}
All the waveform from the the charge cut [] are align with $t_{10}^r$ and average to get the SER.
\subsubsection{Pre-pulse and after-pulse}
Pre-pulses generate due to photons hit on the MCP or the first dynode directly rather than the photocathode\cite{JUNOMassTesting}. The amplitude of pre-pulses are smaller than normal signal and appear before the main pulse. This ratio is related to the intensity of light source.
% waveform analysis
Afterpulse are generated due to the ionization of gaseous impurities between the cathode and first dynode when photo-electrons go through\cite{Coates_1973}. These ions hit back on the photocathode and generate electrons. $H^+,He^+,O^+$ are the major ions contributing to afterpulse\cite{Coates_1973}. Due to these ions are heavy than electron, the travel time is in the scale of \si{us}\footnote{The velocity of ions is about \SI{1000}{km/s} and size of PMT is about \SI{0.1}{m}, thus the transit time is about \SI{0.1}{us}}. The after pulse is calculated in a window \SIrange{300}{10000}{ns} after the main pulse.


Afterpulse is categorized into several kinds. Fig~\ref{fig:afterpulse} indicates serveral typical after-pulse peaks in time around \SI{1000}{ns} with ratio about xx\%. Considering the different mass of ions, these peaks originates from xx, xx.
\subsubsection{relative photon detection efficiency}
To measure PDE, the intensity of light and light allocation ratio of each channel need to be calibrated. For example JUNO fixed one reference PMT and other reference PMTs are circulated through all channels \cite{Wonsak_2021}. A new method is designed to reduce the number of reference PMT and combine all test runs to do calibration.  

Note $n,j,k$ ($n=0,...,N-1, j=0,...,J-1, k=0,...,K-1$) is the indicator of test run, channel of splitter and PMT. Intensity of light is $I_n$, light allocation ratio is $\alpha_j$ and photon detection efficiency is $\eta_k$. Assume $\alpha_j$s are stable among different test runs. The trigger rate of nth run, kth PMT in jth channel is
\begin{equation}
    \label{equ:pderate}
    R_{njk}=I_n\alpha_j\eta_k
\end{equation}
For convenience, 0th PMT is the reference PMT. Note $\alpha_j^0=\frac{\alpha_j}{\alpha_0}$, $\eta_k^0=\frac{\eta_k}{\eta_0}$, $I_n^0=I_n\alpha_0\eta_0$, Equ~\eqref{equ:pderate} can be transfer to Equ~\eqref{equ:pdelograte}
\begin{equation}
    \label{equ:pdelograte}
    \mathrm{log}(R_{njk})=\mathrm{log}(I_n^0)+\mathrm{log}(\alpha_j^0)+\mathrm{log}(\eta_k^0)
\end{equation}
Equ~\eqref{equ:pdelograte} is a equation set with $NK$ equations. $I_n^0,\alpha_j^0,\eta_k^0$ are the unknown $N+K+J-2$ parameters, which can stacked into an array $X = C(\mathrm{log}(I_n^0), \mathrm{log}(\alpha_j^0),\mathrm{log}(\eta_k^0))$ ($n=0,...,N-1, j=1,...,J-1, k=1,...,K-1$). 
\begin{equation}
    \mathrm{log}(R_{njk})=DX
\end{equation}
in which $D=[D_I,D_\alpha, D_\eta]$ is a sparse matrix, $D_I[{njk},n]=1,D_\alpha[{njk},j-1]=1, D_\eta[{njk},k-1]=1$. 
In Appendix~\ref{sec:solution} there exist solution only when the  equation set