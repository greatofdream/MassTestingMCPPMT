\sisetup{separate-uncertainty=true}
\sisetup{multi-part-units=single}
\section{Methods and results}
\label{Method}
% The setup of the experiment and preanalysis are discussed in Section~\ref{sec:laserstage}. Analysis method and results of parameters are introduced in the following sections.
\subsection{Preanalysis}
\label{sec:laserstage}

The laser intensity is adjusted to the level of $1/20$ occupancy to obtain single PE events. The window size $T_{\mathrm{wave}}$ is \SI{10400}{ns} to include all the possible after-pulses (see Section~\ref{sec:afterpulse}). The rising edge of the trigger waveform from the laser system is linearly interpolated to get the half-height time $t_{\mathrm{trig}}$ at about \SI{200}{ns}, as shown in Fig.~\ref{fig:triggertime}.
\begin{figure}[!htbp]
    \centering
    \includegraphics[width=\LF\textwidth]{figures/method/triggerwave.pdf}
    \caption{An example of PMT and trigger signals. The orange line is the trigger waveform, with its high and low voltages indicated by two black horizontal dashed lines. The green cross is the interpolated trigger time $t_{\mathrm{trig}}$. The solid blue line is the PMT waveform with a PE pulse. The horizontal violet dotted line is the voltage threshold for calculating the baseline shown by the blue horizontal dashed line. The black horizontal dash lines intersect 10\%, 50\% and 90\% of (downward) rising and falling edges. The red and yellow vertical dashed lines are for the 10\% rising edge and the pulse peak respectively.}
    \label{fig:triggertime}
\end{figure}


In the preanalysis, we select a preliminary window $[t_{\mathrm{trig}},600\,\mathrm{ns}]$ where dark noises (\SI{\sim 10}{kHz} in Section~\ref{sec:dcr}) and laser pulses are expected to contribute 0.004 and 0.05 counts on average. The peak time $t_p$ is the minimum position in each window, as shown in Fig.~\ref{fig:triggertime}.
%Because the maximum pulse is selected in each waveform, the expected charge distribution is the distribution of $\max(C_n)$ ($n=1,...,N$), in which $C_n$ is the charge of the nth PE and $N$ is the number of PE in a waveform.



The baseline not being zero, we estimate it from the sidebands $[-200\,\mathrm{ns},-10\,\mathrm{ns}]$ and $[100\,\mathrm{ns},200\,\mathrm{ns}]$ relative to $t_p$. To remove potential additional pulses in the sidebands, as the horizontal violet dotted line in Fig.~\ref{fig:triggertime} we define a voltage threshold from a rough estimation of white noise and cut off additional \SI{10}{ns} around each over-threshold time interval. The baseline $\mu_b$ is estimated as the average of the residual sidebands. The peak height $V_p$ of a pulse is the difference between $\mu_b$ and the minimum voltage.

Over all the waveform samples, a Gaussian $f(t;t_0,\sigma_{t0}^2)$ is fitted to the distribution of $t_p-t_{\mathrm{trig}}$ of pulses whose $V_p$ exceeds \SI{5}{ADC}. We define a new candidate window, about \SI{30}{ns} long, as $[t_0-5\sigma_{t0}, t_0+5\sigma_{t0}]$ to calculate new $t_p$ and $V_p$ by repeating the above procedures.  With such a candidate window the dark counts reduced by 10, we shall conduct all the analysis in the following sections except pre/after-pulses.

\subsection{Single-PE charge spectrum and resolution}
\label{sec:noisepeak}

Considering the rise and fall time distributions, the charge $Q$ of a pulse is the summation of the baseline-subtracted voltages in a time window $[\SI{-10}{ns}, \SI{75}{ns}]$ relative to $t_p$ as illustrated in the pink region of Fig.~\ref{fig:triggertime}. The input impedance being \SI{50}{\Omega}~\cite{CAENV1751}, the charge in the unit of Coulomb is $Q/\SI{50}{\Omega}$.

Such $Q$ in Fig.~\ref{fig:triggercharge} represents the charge of a single PE with negligible multi-PE contributions due to the low occupancy. A long tail is evident in the single-PE charge distribution, also found in the NNVT 20-inch MCP-PMTs by the JUNO collaboration~\cite{JUNOMassTesting}. Zhang~et~al.~\cite{JUNOLongtail} proposed a phenomenological parameterization without dedicated consideration of the multiplication process of the PEs. Our future publications will discuss the physical model and solution of the long tail.

To describe the peak shape of the $Q$ distribution, a Gaussian function $f(Q;Q_0,\sigma^2_{Q_0})$ is fitted to the interval $[0.65Q_0, 1.35Q_0]$ via the modified least-square (MLS)~\cite{Cowan1998StatisticalDA} as the red line in Fig.~\ref{fig:triggercharge}. To remove the influence of the pedestal and describe the long tail~\cite{JUNOLongtail}, pulses with $V_p>\SI{3}{ADC}$ and $Q>0.25Q_0$ are selected to calculate the mean $\overline{Q}$ and sample variance $s^2_{Q}$ of $Q$.
% The $V_p$ distribution with \SI{1}{ADC} bin width in Fig.~\ref{fig:triggerpeak} 
The two cuts are complementary to exclude some noises with small $V_p$ but large $Q$~(Fig.~\ref{fig:triggercharge}).

\begin{figure}[!htbp]
    \centering
    \begin{subfigure}[b]{\SF\textwidth}
        \includegraphics[width=\textwidth]{figures/method/triggercharge.pdf}
        \caption{}%PM
        \label{fig:triggercharge}
    \end{subfigure}
    \begin{subfigure}[b]{\SF\textwidth}
        % \includegraphics[width=\textwidth]{figures/method/triggerpeak.pdf}
        % \caption{}%PM
        % \label{fig:triggerpeak}
        \includegraphics[width=\textwidth]{figures/result/gainres.pdf}
        \caption{}
        \label{fig:totalchargeCompare}
    \end{subfigure}
    \caption{(\subref{fig:triggercharge}) The long-tailed single-PE charge distribution of an MCP-PMT. The entries around zero are waveforms with no signal. The vertical blue dashed line is the pedestal cut. The orange histogram is the selected waveforms with peak-height cut ($V_p>\SI{3}{ADC}$). The pink and green areas are the fit intervals for the peak and valley parameters. (\subref{fig:totalchargeCompare}) The charge and resolution ratios show the effect of the long tail. The point and crosses represent the reference and MCP PMTs, respectively.
    }
\end{figure}

The gains of the main peak and the entire sample are ${Q_0}/({\SI{50}{\Omega}} e) \approx \num{e7}$ and ${\overline{Q}}/(\SI{50}{\Omega} e)$, $e$ being the charge of an electron. The relative \emph{peak} and \emph{sample resolutions} $\nu_0$ and $\nu$ are defined as ${\sigma_{Q_0}}/{Q_0}$ and ${\sqrt{s^2_{Q}}}/{\overline{Q}}$.  Shown in Fig.~\ref{fig:totalchargeCompare}, $\overline{Q}$ is about 1.8 times $Q_0$ for the MCP-PMTs, in agreement with Zhang~et~al.~\cite{JUNOLongtail}. The long tail worsens the resolution of MCP-PMTs from $\nu_0=\num{0.25 \pm 0.02}$ to $\nu=\num{0.69 \pm 0.03}$, but is less pronounced for the reference dynode PMT.

\subsection{Peak-to-valley ratio}
\label{sec:PV}
A parabolic function is fitted to the valley based on MLS in the interval $[-0.15Q_0, 0.25Q_0]$ relative to the least-counted bin of the histogram between the pedestal and the main peak, as shown in Fig.~\ref{fig:triggercharge}. The \emph{valley count} $N_v$ is defined as the minimum of the parabola and \emph{peak count} $N_p$ is the maximum of the Gaussian described in Section~\ref{sec:noisepeak}. The peak-to-valley ratio~(P/V) ${N_p}/{N_v}$ shows the ability to discriminate between electronic noises and a PE signal. The average P/V of MCP-PMTs is about 5.8, significantly higher than that (about 2.4) of the reference PMT.

\subsection{Single electron response}
\label{sec:SER}
As shown in Fig.~\ref{fig:triggertime}, $t^{10}_r$, $t^{50}_r$, $t^{90}_r$ ($t^{10}_f$, $t^{50}_f$, $t^{90}_f$) are the times of interpolated 10\%, 50\%, and 90\% $V_p$ in the rising (falling) edge. We define the rise time $t_r = t^{90}_r - t^{10}_r$, fall time $t_f = t^{10}_f - t^{90}_f$ and full width at half maximum $\mathrm{FWHM} = t^{50}_f - t^{50}_r$ to describe the shape of \emph{single electron response}~(SER).  They are measured to be $t_r = \SI{3.71\pm0.15}{ns}$, $t_f = \SI{15.6\pm1.8}{ns}$ and $\mathrm{FWHM} = \SI{9.07\pm0.63}{ns}$ for the 9 MCP-PMTs.

% \begin{figure}
%     \centering
%         \includegraphics[width=\MF\textwidth]{figures/method/triggerSER.pdf}
%         \caption{A fitting result of a pulse.}%PM
%         \label{fig:triggerser}
    % \begin{subfigure}[b]{\SF\textwidth}
    %     \includegraphics[width=\textwidth]{figures/result/tausigma.pdf}
    %     \caption{}%PM
    %     \label{fig:sigmaCompare}
    % \end{subfigure}
    % \caption{(a)  (b) $\tau$ versus $\sigma$ of 9 MCP-PMTs.}
% \end{figure}
To get a smooth SER, we select signals with $V_p>\SI{3}{ADC}$, $Q \in [0.5Q_0, 1000\mathrm{ADC\cdot ns}]$ and $\mathrm{FWHM} \in [2\,\mathrm{ns}, 15\,\mathrm{ns}]$ to exclude the noise and large pulses. An \emph{exGaussian} distribution $f^N(t;\mu_{\mathrm{SER}},\sigma_\mathrm{SER}^2)\otimes f^{\mathrm{Exp}}(t;1/\tau_\mathrm{SER})$~\cite{Luo:2022xrd} is used to fit the SER% as shown in Fig.~\ref{fig:triggerser}
, in which $\mu_{\mathrm{SER}}$ is the time offset of the pulse, $\sigma_{\mathrm{SER}}$ and $\tau_{\mathrm{SER}}$ model the shape of SER. They are measured to be $\tau_{\mathrm{SER}} = \SI{7.2\pm1.0}{ns}$ and $\sigma_{\mathrm{SER}} = \SI{1.62\pm0.06}{ns}$ for the 9 MCP-PMTs.

\subsection{Transit time spread}
\label{sec:TTS}
The PEs from the photocathode drift to the MCP, as shown in Fig.~\ref{fig:mcpelectron}. The drifting electric field and the PE trajectory are simulated by a simplified model consisting of a cathode at \SI{0}{V}, a focusing electrode at \SI{480}{V} and an MCP at \SI{528}{V}. The PEs from the top of the photocathode with \SI{0}{eV} and \SI{3}{eV} kinetic energies have drift times of about \SI{21}{ns} and \SI{18}{ns}, respectively. A PE entering an MCP channel is multiplied to be an observable pulse, while that hitting the surface of the MCP gets scattered inelastically into several secondary electrons or elastically into one single electron~\cite{Furman}. The scattered electrons drift in the electric field until entering the MCP channels finally to give delayed pulses~\cite{KM3NetTesting}. Multiple secondary electrons with different kinetic energies may cause two or more pulses due to different drift times, as shown in Fig.~\ref{fig:triggerTT2pulse}.

\begin{figure}[!htbp]
    \centering
    \includegraphics[width=\MF\textwidth]{figures/method/MCPelectron.pdf}
    \caption{The PEs drift from the photocathode to the MCP and get amplified or scattered.}%PM
    \label{fig:mcpelectron}
\end{figure}

The transit time (TT) of a PE is the time needed to travel from the photocathode to the anode, mainly composed of the drift and multiplication times. However, the absolute TT is hard to measure. A relative $\mathrm{TT}$ is defined as the time difference between the trigger signal $t_{\mathrm{trig}}$ and 10\% of the rise time of a PE pulse $t_r^{10}$. The $\mathrm{TT}$ distribution of the MCP-PMTs contains slowly rising and falling edges on both sides of the peak, as shown in Fig.~\ref{fig:triggerTTSLog}. The rising edge is due to the PEs with larger kinetic energies, while the falling one consists of secondary electrons with longer drift times~\cite{longtail}.

Delayed pulses are searched in the interval $[t_0+20\,\mathrm{ns},t_0+80\,\mathrm{ns}]$ to separate them from the main pulses in $[t_0-5\,\mathrm{ns},t_0+5\,\mathrm{ns}]$. The blue histogram in Fig.~\ref{fig:triggerTTlatepulse} is the distribution of the delayed pulses, and the filled one is for those with the main pulses in the same waveform, an example demonstrated in Fig.~\ref{fig:triggerTT2pulse}. The sharp difference between them at about \SI{40}{ns} after the main peak, twice the drift time of PEs from the cathode to the MCP, reasonably illustrates that an elastically scattered electron cannot appear together with a main pulse in a waveform.

\begin{figure}[!htbp]
    \centering
    \begin{subfigure}[t]{\SF\textwidth}
        \includegraphics[width=\textwidth]{figures/method/triggerDoublePulse.pdf}
        \caption{}%PM
        \label{fig:triggerTT2pulse}
    \end{subfigure}
    \begin{subfigure}[t]{\SF\textwidth}
        \includegraphics[width=\textwidth]{figures/method/triggerDelayedPulse.pdf}
        \caption{}%PM
        \label{fig:triggerTTlatepulse}
    \end{subfigure}
    \caption{(a) Double-pulse example. The first pulse falls in the candidate window defined in Section~\ref{sec:laserstage}. (b) The green and blue histograms are the main and delayed pulse distributions, respectively. The filled histograms are waveforms containing the main and delayed pulses simultaneously.}
\end{figure}

The main and early components are modeled with Gaussians $N_1f_1^N(t;\mu_{\mathrm{TT}},\sigma_{\mathrm{TT}}^2)$ and $N_2f_2^N(t;\mu_K,\sigma_K^2)$, the subscript $K$ standing for PEs with high kinetic energies. Considering the exponential distribution of kinetic energies of secondary electrons~\cite{Furman,SecondElectron}, $f^\mathrm{Exp}(t;1/\tau_S)$ is suitable to model the delayed component.  A more comprehensive measurement is ongoing.  We artificially add a constant $b_S$ and a translation of $\mu_{TT} + 3\sigma_{TT}$ to fit the data. As shown in Fig.~\ref{fig:triggerTTSLog}, the \SI{0.5}{ns}-binned histogram of $\mathrm{TT}$ is fitted by

\begin{equation}
    \begin{aligned}
        B&+N_1f_1^N(t;\mu_{\mathrm{TT}},\sigma_{\mathrm{TT}}^2)\\
        &+N_2f_2^N(t;\mu_K,\sigma_K^2)\\
        &+b_SH(\mu_{\mathrm{TT}}+3\sigma_{\mathrm{TT}})+N_Sf^{\mathrm{Exp}}\left(t-(\mu_{\mathrm{TT}}+3\sigma_{\mathrm{TT}});\frac{1}{\tau_S}\right)
    \end{aligned}
\end{equation}
in which $H$ is the Heaviside function to limit the definition domain of the delayed component, and $B$ is the constant dark noise rate. The early and exponential components can be omitted in a rough simulation due to their small ratios in the 9 MCP-PMTs: $N_2/N_1 = \num{0.042\pm0.013}$, $N_S/N_1 = \num{0.014\pm0.002}$ and $2b_S\sigma_{\mathrm{TT}}/N_1 = \num{0.00028\pm0.00005}$.  $\sigma_K$, $\tau_S$ and $\mu_{\mathrm{TT}}-\mu_K$ are fitted to be $\SI{1.4 \pm 0.3}{ns}$, $\SI{1.1\pm 0.1}{ns}$ and $\SI{3.2\pm 0.2}{ns}$. TTS ($\SI{1.74\pm0.05}{ns}$) is defined as FWHM$=2\sqrt{2\ln 2}\sigma_{\mathrm{TT}}$~\cite{HAMAMATSUManual} representing the timing resolution. The charge and TT seem to have some correlation in Fig.~\ref{fig:triggerTTS2d}, the long tail evident in charge distribution.

\begin{figure}[!htbp]
    \centering
    \begin{subfigure}[t]{\SF\textwidth}
        \includegraphics[width=\textwidth]{figures/method/triggerTTSLog.pdf}
        \caption{}%PM
        \label{fig:triggerTTSLog}
    \end{subfigure}
    \begin{subfigure}[t]{\SF\textwidth}
        \includegraphics[width=\textwidth]{figures/method/triggerTTS2d.pdf}
        \caption{}%PM
        \label{fig:triggerTTS2d}
    \end{subfigure}
    \caption{(a) The top pad is the distribution of TT with the y-axis in the logarithmic scale. The black dashed lines from left to right are early, main and delayed components with a dark noise pedestal. The bottom pad is the residual between data and fit in linear scale. (b) The 2D distribution of TT and charge with the colorbar in logarithmic scale.}
\end{figure}

\subsection{Dark count rate}
\label{sec:dcr}
The dark noise mimicking PEs mainly comes from the spontaneous thermionic electrons emitted from the photocathode~\cite{KM3NetTesting}. The dark count rate (DCR) is ${N^{\mathrm{noise}}}/({N^{\mathrm{hit}}T_{\mathrm{DCR}}})$, in which $N^{\mathrm{noise}}$ is the noise count in the interval of $[\SI{-300}{ns},\SI{-150}{ns}]$ relative to the main pulse with $T_{\mathrm{DCR}}=\SI{150}{ns}$ and $N^{\mathrm{hit}}$ is the number of waveforms with main pulses. The DCR of 9 MCP-PMTs is $5.3\pm\SI{1.7}{kHz}$ at room temperature.

\subsection{Pre-pulse and after-pulse}
\label{sec:afterpulse}
Generated from photons hitting the MCP or the first dynode directly rather than the photocathode, \emph{pre-pulses} appear about tens of nanoseconds earlier with smaller amplitudes~\cite{JUNOMassTesting}.

Ions such as \ce{H^+}, \ce{He^+} and \ce{O^+} produced from gaseous impurities in the vacuum bulb by the PEs drift back to the photocathode, generate new electrons and further \emph{after-pulses}~\cite{JUNOMassTesting,Coates_1973}. The delay times of after-pulses are proportional to the square root of the mass-to-charge ratios of the ions~\cite{XENON1TTesting,Coates_1973,afterpulseTime}. %Considering electric field and $\frac{M}{Z}$ of ions, the travel time is in the scale of \si{us}.
\begin{figure}
    \centering
    \includegraphics[width=\LF\textwidth]{figures/method/triggerAfterpulseSchema.pdf}
    \caption{An example waveform for searching pre-pulses and after-pulses. Green and red vertical dashed lines are the trigger time and 10\% of the rising edge of the main pulse $t_r^{10}$, respectively. The gray and orange regions are the intervals for searching after-pulses and pre-pulses.}
    \label{fig:afterpulseSchema}
\end{figure}

The pre-pulses and after-pulses are searched from \SI{10}{ns} before and \SI{200}{ns} after the main pulse. The 10\% rise time $t_r^{10}$ and charge $Q$ of the after-pulse and pre-pulse are calculated in the $[-\SI{10}{ns},\SI{75}{ns}]$ window relative to the peak position, as shown by the violet area in Fig.~\ref{fig:afterpulseSchema}.

The ratios of pre-pulses $R_{\mathrm{pre}}$ and after-pulses $R_{\mathrm{after}}$ are calculated in the time interval [\SI{-100}{ns},\SI{-10}{ns}] ($T_{\mathrm{pre}}=90$\,ns) and [\SI{200}{ns},\SI{9800}{ns}] ($T_{\mathrm{after}}=9600$\,ns) relative to the main pulses respectively as the following equations

\begin{align}
    R_{\mathrm{pre}} = \frac{N^{\mathrm{pre}}}{N^\mathrm{hit}} - \mathrm{DCR}\cdot T_{\mathrm{pre}}\\
    R_{\mathrm{after}} = \frac{N^{\mathrm{after}}}{N^\mathrm{hit}} - \mathrm{DCR}\cdot T_{\mathrm{after}}
\end{align}
in which $N^{\mathrm{pre}}$ and $N^{\mathrm{after}}$ are the number of pre-pulses and after-pulses.  The small $R_{\mathrm{pre}}$ ($3\times10^{-5}\pm1.2\times10^{-4}$) is dominated by DCR at our low-occupancy setup, and $R_{\mathrm{after}}$ of 9 MCP-PMTs is $0.009\pm0.006$.

\begin{figure}[!htbp]
    \centering
    \begin{subfigure}[t]{\LF\textwidth}
        \includegraphics[width=\textwidth]{figures/method/triggerAfterpulse1d.pdf}
        \caption{}%PM
        \label{fig:afterpulse1d}
    \end{subfigure}
    \begin{subfigure}[t]{\LF\textwidth}
        \includegraphics[width=\textwidth]{figures/method/triggerAfterpulse2d.pdf}
        \caption{}
        \label{fig:afterpulse2d}
    \end{subfigure}
    \caption{(a) An example of time distributions of pre-pulses (orange) and after-pulses (blue) for an MCP-PMT, the green line being the Gaussian fits. The blank area around the \SI{0}{ns} is the main pulses not shown in this figure. (b) Charge versus time distribution of pre-pulses and after-pulses. The bright horizontal band at about 100\,$\mathrm{ADC}\cdot \mathrm{ns}$ mainly contains dark noise. The black line is the mean charge of pre/after-pulses in each time bin.}
\end{figure}
% waveform analysis

The distribution of the delay time from $t_r^{10}$ of the main to after-pulse in Fig.~\ref{fig:afterpulse1d} indicates five typical peaks or structures at around \SI{300}{ns}, \SI{400}{ns}, \SI{600}{ns}, \SI{1200}{ns} and \SI{1700}{ns}, time ratio being about $1:\sqrt{2}:\sqrt{4}:\sqrt{16}:\sqrt{32}$. These groups may originate from \ce{H^+}, \ce{H_2^+}, \ce{He^{+}}, \ce{CH_4^+}, and \ce{N_2^+} or \ce{O_2^+}. XENON1T~\cite{XENON1TTesting} and XMASS~\cite{Abe_2020} gave the same assumptions for the first peak. Similar works by JUNO~\cite{Zhao:2022gks} and KM3NeT~\cite{KM3NetTesting} claimed the first peak to be \ce{H_2^{+}}.

We use five Gaussians $\sum_{i=1}^{5}{A_if_i^{\mathrm{AP}}(t;t_i,\sigma_i^2)}$ to model the five groups after subtracting DCR, in which $A_i$, $t_i$, and $\sigma_i$ are the amplitudes, times, width of each after-pulse group (Table.~\ref{tab:afterpulse}). The amplitudes of the groups vary greatly for different MCP-PMTs in Fig.~\ref{fig:afterpulsePeak}. Besides, we suggest roughly modeling the time and charge distribution of unexplained slow undulating structures contributing about half of the after-pulses on average in the time interval $[\SI{2000}{ns},\SI{7000}{ns}]$, as shown in Fig.~\ref{fig:afterpulse1d}, with the uniform time and the single PE charge distribution.

\begin{table}
    \centering
    \caption{Parameters of after-pulse of MCP-PMTs}
    \label{tab:afterpulse}
    \begin{threeparttable}
        \begin{tabular}{c|c|c|c|c|c}
            \hline
            &1st group&2nd group&3rd group&4th group&5th group\\
            \hline
            $t_i$/ns&310$\pm$8&435$\pm$27&593$\pm$24&1201$\pm$41&1725$\pm$25\\
            $A_i/N^{\mathrm{hit}}$(/1E3)&1.2$\pm$0.7&0.3$\pm$0.2&0.3$\pm$0.3&0.8$\pm$0.6&1.5$\pm$0.6\\
            $\sigma_i$/ns&15$\pm$6&40$\pm$16 &32$\pm$12&50$\pm$26&68$\pm$23\\
            $Q_i/Q_0$&32$\pm$3&32$\pm$3\tnote{1}&20$\pm$3&17$\pm$3&14$\pm$1\\
            $\sigma_{Q_i}/Q_0$&12$\pm$1&12$\pm$1\tnote{1}&21$\pm$7&9$\pm$1&8$\pm$1\\
            \hline
        \end{tabular}
        \begin{tablenotes}
            \footnotesize
            \item[1] The charge parameters of the 2nd group inherit from the 1st group.
        \end{tablenotes}
    \end{threeparttable}
\end{table}

\begin{figure}[!htbp]
    \centering
    \begin{subfigure}[t]{\SF\textwidth}
        \includegraphics[width=\textwidth]{figures/result/afterpulse.pdf}
        \caption{}%PM
        \label{fig:afterpulsePeak}
    \end{subfigure}
    \begin{subfigure}[t]{\SF\textwidth}
        \includegraphics[width=\textwidth]{figures/result/afterpulseQ.pdf}
        \caption{}%PM
        \label{fig:afterpulseQ}
    \end{subfigure}
    \caption{(a) Time and relative amplitudes of after-pulse groups of 9 MCP-PMTs. (a) Time and mean charge of after-pulse groups of 6 MCP-PMTs.}
\end{figure}

%The mean charge in each time bin shown as the black line. 
The after-pulse groups contain large charge signals, as shown in Fig.~\ref{fig:afterpulse2d}. The charge distributions in $[t_i-3\sigma_i,t_i+3\sigma_i]$ of each after-pulse group except for the 2nd one are scaled to the same dark noise count in Fig.~\ref{fig:afterpulsecharge}. We suggest using the same charge parameters with the 1st group for the 2nd group due to overlap with the 1st. The low charge area is consistent with dark noise, and the large charge events contributed from after-pulses are fitted with a Gaussian $f^{AP_Q}_i(Q;Q_i,\sigma_{Q_i}^2)$, in which $Q_i$ and $\sigma_{Q_i}$ are the charge and spread of each after-pulse group (Table.~\ref{tab:afterpulse}). The mean charge of each group without the 2nd one of six high statistics PMTs in Fig.~\ref{fig:afterpulseQ} shows a negative correlation between charge and delay time. The charge distribution of 3rd low statistics group cannot be fitted well and therefore $\sigma_{Q_3}$s are large. 

\begin{figure}[!htbp]
    \centering
    \includegraphics[width=\textwidth]{figures/method/triggerafterpulseCharge.pdf}
    \caption{The charge distribution of after-pulse groups of an example MCP-PMT. The blue histogram is the distribution of dark noise. The violet points are the peaks of fit curves.}
    \label{fig:afterpulsecharge}
\end{figure}

\subsection{Relative photon detection efficiency}
\label{sec:PDE}
 %For example, JUNO fixed one reference PMT to calibrate the light intensity and other reference PMTs are circulated through all channels to calibrate the light allocation ratio~\cite{Wonsak_2021}.
A regression method is developed to combine the light-source calibration and PDE measurements simultaneously. Note $I_n$ to be the light intensity of the $n$th run, $\alpha_j$ to be the light allocation ratio of the $j$th splitter channel (out of four and assumed to be stable across runs), $\eta_k$ to be the PDE of the $k$th PMT (out of one reference dynode and nine MCP PMTs). The PE counts in each waveform obey Poisson distribution $\pi(I_n\alpha_j\eta_k)$.

For convenience, the index of the reference PMT is set to 0. Note $\alpha_j^0\equiv\alpha_j/{\alpha_0}$, $\eta_k^0\equiv\eta_k/{\eta_0}$, $I_n^0\equiv I_n\alpha_0\eta_0$.  The hit rate $R_{njk}$ of the $k$th PMT at the $j$th channel in the $n$th run is

\begin{equation}
    \label{equ:linkfunction}
    R_{njk}=1-\exp\left(-I_n\alpha_j\eta_k\right)=1-\exp\left(-e^{\log{I_n^0}+\log{\alpha_j^0}+\log{\eta_k^0}}\right).
\end{equation}

The number of hit waveforms $N^{\mathrm{hit}}_{njk}$ of the $k$th PMT in the $n$th run with the $j$th channel obeys Binomial distribution $B(N^{\mathrm{hit}}_{njk};R_{njk},N^t_{njk})$, in which $N^t_{njk}$ is the total number of waveforms by the laser trigger. The likelihood is therefore

\begin{equation}
    \label{equ:likelihood}
    \mathcal{L}=\prod_{njk}{R_{njk}^{N^\mathrm{hit}_{njk}}(1-R_{njk})^{N^t_{njk}-N^{\mathrm{hit}}_{njk}}}.
\end{equation}

Eqs.~\eqref{equ:linkfunction} and \eqref{equ:likelihood} defines a \emph{Binomial regression} with \emph{complementary log-log} link function~\cite{glm}, with $\log{\eta_k^0}$, $\log{\alpha_j^0}$ and $\log{I_n^0}$ as parameters. The relative PDEs $\eta_k^0$ of MCP-PMTs are calculated from the regression results to be about $1.71$, significantly higher than the reference PMT.  We could attribute the PDE to the improvements on both quantum and collection efficiencies of the MCP-PMTs. 
