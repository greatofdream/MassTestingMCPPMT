\section{Introduction}
% Experiment Introduction and Importance of resolution for Low PE
The Jinping Neutrino Experiment (JNE) of multi-hundred-tons is a large-volume water-based liquid scintillator experiment with \SI{2400}{m} rock overburden \cite{LetterJNE2017} which is under construction. Its main goal is to accurately measure the spectrum of MeV-scale neutrinos, including solar neutrinos, geo-neutrinos, and supernova neutrinos \cite{LetterJNE2017}. % PMT Introduction
The photomultiplier tube (PMT) \cite{HAMAMATSUManual}, which transforms a photon into a single photoelectron (PE), and then to the measurable electric signal, is often used to detect few photons in water Cherenkov detectors \cite{SNO,SuperK} and liquid scintillator detectors \cite{KamLAND,JUNO:2015zny}. The details of the energy spectrum of low energy neutrinos are determined by the energy resolution, which is dependent on the coverage of light collection and photon detection efficiency of PMTs. The price of PMTs also plays an important role for the coverage of light collection during design. The direction information from Cherenkov photons, which is centered in less than \SI{1}{ns} for electrons less than \SI{30}{MeV}, provides the ability to suppress the background in the solar neutrinos and supernova relic neutrinos measurements for the Cherenkov scintillation detector \cite{Guo_2019}. The spectrum of Cherenkov light spread from \SI{300}{nm} to \SI{700}{nm} \cite{Luo:2022xrd}. The uncertainty from dispersion is estimated as $R\frac{\delta v_C}{v_C^2}$, in which $R$, $\delta v_C$, and $v_C$ are the radius of the detector, the uncertainty of velocity, and center value of the velocity of Cherenkov light. Considering the uncertainty from dispersion is about \SI{1.3}{ns} for \SI{10}{m} scale detectors\footnote{The values are estimated as \SI{10}{m}, \SI{5}{mm/ns} and \SI{192}{mm/ns} from \cite{Luo:2022xrd}.}, the short time scale of Cherenkov light requires the deviation of transit time to be at the scale of \SI{1}{ns}.

Instead of conventional discrete dynodes, the micro-channel plate (MCP) PMT use MCP to multiply photoelectron inside the micro-channels, which offers faster time response and high gain in a compact size \cite{HAMAMATSUManual}. The new type of 8-inch MCP-PMT (GDB-6082 \cite{GDB-6082}) meeting above requirements including good cost performance, which produced by North Night Vision Science \& Technology (Nanjing) Research Institute Co. Ltd. (NNVT), will be used in JNE. The new type of MCP-PMTs have high photon detection efficiency, which is the product of collection efficiency and quantum efficiency. This work is to confirm the superiority of photon detection efficiency (PDE) and boost on energy resolution of the new type of MCP-PMTs. This also is the first measurement for the new type of PMTs.

% Testing of Other Experiments
Measurements for the 20-inch MCP-PMT from NNVT, which has a similar structure to 8-inch MCP-PMT, were done by JUNO and confirmed that the average PDE is about 28\% \cite{JUNOMassTesting}. In addition, several other recent PMT characterization tests were reviewed and referenced. Gain, single PE resolution, PDE, transit time spread (TTS), and dark count rate (DCR) are the most important parameters measured in recent different experiments. For example, Daya Bay tested 8-inch dynode PMTs (9354KB, R5912, XP1806) \cite{DayaBayTesting}, Double Chooze used 10-inch dynode PMTs (R7081) \cite{DoubleChoozeTesting}, LHAASO used 8-inch dynode PMTs (CR365-02-1) \cite{LHAASOTesting}, HyperKamiokande tested 20-inch dynode PMTs (R12860) and MCP-PMTs (GDB-6203) \cite{HyperKTesting}, KM3Net used 3-inch dynode PMTs (R12199-02) \cite{KM3NetTesting}, XENON1T and XENONnT tested 3-inch dynode PMTs (R11410-21) \cite{XENON1TTesting}\cite{XENONnTTesting}, IceCube tested 3-inch PMTs (R12199-01) \cite{IceCubeTesting}.

% The structure of paper
This work concentrated on the characteristics of MCP-PMTs in the low photon intensity condition. The setup of the testing system and facility is introduced in sec.~\ref{SetUp}. Detailed analysis methods and results, including Gain, single PE resolution, PDE, TTS, DCR, time characteristics of single electron response (SER), pre-pulse, and after-pulse are described in sec.~\ref{Method}. The boost for energy resolution is discussed in sec.~\ref{Result}. Finally, a summary is given in sec.~\ref{Summary}.
