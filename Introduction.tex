\section{Introduction}
% Experiment Introduction and Importance of resolution for Low PE
The Jinping Neutrino Experiment (JNE) of 500t is a large-volume water based liquid scintillator experiment with \SI{2400}{m} rock overburden\cite{LetterJNE2017} which is under construction. Its main goal is to accurately measure the spectrum of MeV-scale neutrinos, including solar, Geo and Supernova neutrinos\cite{LetterJNE2017}.  The energy resolution for low energy determines the details of energy spectrum of neutrinos, which is dependent on the coverage of light collection and efficiency of photon detection.

% PMT Introduction
Photomultiplier tubes (PMT) is the important instrument for detecting few photons in JNE, which transform photon signal to measurable electric signal. Therefore, the spatial, timing resolution and furthermore energy resolution depend on the characteristics of PMTs, especially the quantum efficiency (QE). 

The new type of 8-inch micro-channel plate(MCP) PMTs(GDB-6082\cite{GDB-6082}) come from Northern Night Vision Technology Ltd.(NNVT) will be used in JNE, which have high collection and quantum effiency. This work is to confirm the superiority and expected boost for resolution of the new type of MCP-PMT, and also is the first measurements for the new type of 8-inch MCP PMT.

% Testing of Other Experiments
Measurements for 20-inch MCP PMT from NNVT, which has similar structure with 8-inch PMT, were done by JUNO and the result shows the PDE is about 28\%\cite{JUNOMassTesting}. In addition, several other PMT characterization testing on dynode PMT were reviewed. Gain, single PE resolution, quantum efficiency, transit time spread(TTS) and dark count rate(DCR) are the most important parameters which measured in recently different experiment. For example, Daya Bay used 8-inch dynode PMTs(9354KB, R5912, XP1806)\cite{DayaBayTesting}, Double Chooze used 10-inch dynode PMTs(R7081)\cite{DoubleChoozeTesting}, LHAASO used 8-inch dynode PMTs(CR365-02-1)\cite{LHAASOTesting}, HyperKamiokande tested 20-inch dynode PMT(R12860 PMT) and MCP PMT(GDB-6203)\cite{HyperKTesting}, KM3Net used 3-inch dynode PMT(R12199-02)\cite{KM3NetTesting}, XENON1T and XENONnT tested 3-inch dynode PMT(R11410-21)\cite{XENON1TTesting}\cite{XENONnTTesting}, IceCube tested 3-inch PMT(R12199-01)\cite{IceCubeTesting}.

% The structure of paper
The setup of testing system and performed procedures will be introduced in sec.~\ref{SetUp}. The testing results of PMTs, including Gain, single PE resolution, relative photon detection efficiency, TTS and DCR, will be presented in sec.~\ref{Result}. Finally, a summary is given in sec.\ref{Summary}.