\section{Introduction}
% Experiment Introduction and Importance of resolution for Low PE
The Jinping Neutrino Experiment under construction is a hundred-ton liquid scintillator detector with dual Cherenkov and scintillation light readout
 at CJPL II with 2400m rock overburden, targeting solar, terrestrial and supernovae neutrinos \cite{LetterJNE2017}.
% PMT Introduction
The photomultiplier tube (PMT) \cite{HAMAMATSUManual}, which transforms a photon into a single photoelectron (PE), and then to the measurable electric signal, is mostly used to detect few photons in water Cherenkov \cite{SNO,SuperK} and liquid scintillator detectors \cite{KamLAND,JUNO:2015zny}. Instead of conventional discrete dynodes, the micro-channel plate (MCP) PMT multiply PE inside the micro-channels of MCP, which offers faster time response and high gain in a compact size \cite{HAMAMATSUManual}.

Precise energy spectra resolution demands affordable PMTs to achieve good photo-coverage with high photon detection efficiency (PDE\footnote{the product of collection efficiency and quantum efficiency}). Cherenkov photons providing directional measurement of solar neutrinos have \SI{1.5}{ns} timing dispersion at 10m scale. The PMTs for this purpose should have time precision of \SI{\sim 1}{ns}.


This work is the first characterization of
 the new type of 8-inch MCP-PMT (GDB-6082 \cite{GDB-6082}) produced by North Night Vision Science \& Technology (Nanjing) Research Institute Co. Ltd. (NNVT) and meeting the above requirements. % Testing of Other Experiments
Similarly structured 20-inch MCP-PMT by NNVT were confirmed by the JUNO collaboration to have an average PDE of 28\% \cite{JUNOMassTesting}.
% important in dark matter experiment
Characterization of gain, single PE resolution, PDE, transit time spread (TTS), dark count rate (DCR) is the key milestone of neutrino and dark matter detectors, such as tests of
 8-inch dynode PMTs (9354KB, R5912, XP1806) at Daya Bay \cite{DayaBayTesting}, tests of 10-inch dynode PMTs (R7081) at Double Chooze \cite{DoubleChoozeTesting}, tests of 8-inch dynode PMTs (CR365-02-1) at LHAASO \cite{LHAASOTesting}, tests of 20-inch dynode PMTs (R12860) and MCP-PMTs (GDB-6203) at HyperKamiokande \cite{HyperKTesting}, tests of 3-inch dynode PMTs (R12199-02) at KM3NeT \cite{KM3NetTesting}, tests of 3-inch dynode PMTs (R11410-21) at XENON1T and XENONnT \cite{XENON1TTesting}\cite{XENONnTTesting}, and tests of 3-inch PMTs (R12199-01) at IceCube \cite{IceCubeTesting}.

% The structure of paper
This work concentrates on the characteristics of 9 MCP-PMTs at low light intensity. The setup of the testing facility is introduced in sec.~\ref{SetUp}. The analysis methods and results of gain, charge resolution, PDE, TTS, DCR, shape of single electron response (SER), pre-pulse, and after-pulse are described in sec.~\ref{Method}. The boost for energy resolution is discussed in sec.~\ref{Result} with a summary in sec.~\ref{Summary}.
