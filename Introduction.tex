\section{Introduction}
% Experiment Introduction and Importance of resolution for Low PE
The Jinping Neutrino Experiment (JNE) under construction is a hundred-ton liquid scintillator detector with Cherenkov and scintillation light readout
 at CJPL II with \SI{2400}{m} rock overburden, targeting solar, terrestrial and supernovae neutrinos~\cite{LetterJNE2017,xu_jinping_2020,xu_innovations_2022,xu_design_2022}.
% PMT Introduction
Photomultiplier tubes~(PMT)~\cite{HAMAMATSUManual} are commonly used to detect individual photons in water Cherenkov~\cite{SNO,SuperK} and liquid scintillator detectors~\cite{KamLAND,JUNO:2015zny}. It converts a photon into a photoelectron~(PE) and then to a measurable electric signal.  Instead of conventional discrete dynodes, the micro-channel plate (MCP) PMT multiplies PEs inside the micro-channels, offering faster time response and high gain in a compact size~\cite{WANG2012113,MCP-PMTworkgroup:2021hoy,HAMAMATSUManual}.

Precise measurement of energy spectra demands affordable PMTs to achieve good photo-coverage with high photon detection efficiency~(PDE\footnote{The product of PE collection and quantum efficiencies}). Cherenkov photons providing a directional measurement of solar neutrinos have \SI{1.5}{ns} timing dispersion at a \SI{10}{m} scale, setting the requirement of time precision to be \SI{\sim 1}{ns}.


This work is the first characterization of
 the new type of 8-inch MCP-PMT (GDB-6082~\cite{GDB-6082}) produced by North Night Vision Science \& Technology (Nanjing) Research Institute Co. Ltd. (NNVT). % Testing of Other Experiments
Similarly structured 20-inch MCP-PMT by NNVT were evaluated by the JUNO collaboration to have an average PDE of 28\%~\cite{JUNOMassTesting}.
% important in dark matter experiment
Characterization of gain, single PE resolution, PDE, transit time spread (TTS) and dark count rate (DCR) is the key step to construct neutrino and dark matter detectors, such as tests of
 8-inch dynode PMTs (9354KB, R5912, XP1806) at Daya Bay~\cite{DayaBayTesting}, 10-inch dynode PMTs (R7081) at Double Chooz~\cite{DoubleChoozeTesting}, 8-inch dynode PMTs (CR365-02-1) at LHAASO~\cite{LHAASOTesting}, 20-inch dynode PMTs (R12860) and MCP-PMTs (GDB-6203) at HyperKamiokande~\cite{HyperKTesting}, 3-inch dynode PMTs (R12199-02) at KM3NeT~\cite{KM3NetTesting}, 3-inch dynode PMTs (R11410-21) at XENON1T~\cite{XENON1TTesting} and XENONnT~\cite{XENONnTTesting}, and tests of 3-inch PMTs (R12199-01) at IceCube~\cite{IceCubeTesting}.

% The structure of paper
This work concentrates on the characteristics of 9 MCP-PMTs at low light intensity. The setup of the testing facility is introduced in Section~\ref{SetUp}. The analysis methods and results of gain, charge resolution, PDE, TTS, DCR, shape of single electron response~(SER), pre-pulse and after-pulse are described in Section~\ref{Method}. The boost for energy resolution is discussed in Section~\ref{Result} with a summary in Section~\ref{Summary}.
