\section{Introduction}
% Experiment Introduction and Importance of resolution for Low PE
The Jinping Neutrino Experiment (JNE) of multi-hundred-tons is a large-volume water-based liquid scintillator experiment with \SI{2400}{m} rock overburden \cite{LetterJNE2017} which is under construction. Its main goal is to accurately measure the spectrum of MeV-scale neutrinos, including solar, Geo, and Supernova neutrinos \cite{LetterJNE2017}. The details of the energy spectrum of neutrinos of low energy is determined by the energy resolution, which is dependent on the coverage of light collection and photon detection efficiency.

% PMT Introduction
To detect few photons, the photomultiplier tube (PMT), which transforms photon signal to measurable electric signal, is used in JNE. Therefore, the spatial, timing resolution, and furthermore energy resolution of a detector depend on the characteristics of PMTs, especially the PDE.

The new type of 8-inch micro-channel plate PMT (MCP-PMT) (GDB-6082 \cite{GDB-6082}), which comes from Northern Night Vision Technology Ltd. (NNVT), will be used in JNE. The new type of MCP-PMT have high collection and quantum effiency\footnote{Photon detection efficiency is the product of collection effiency and quantum effiency.}. This work is to confirm the superiority of PDE and boost on energy resolution of the new type of MCP-PMT, and also is the first measurement for the new type of 8-inch MCP-PMT.

% Testing of Other Experiments
Measurements for 20-inch MCP-PMT from NNVT, which has a similar structure with 8-inch MCP-PMT, were done by JUNO and confirmed that the average PDE is about 28\% \cite{JUNOMassTesting}. In addition, several other recent PMT characterization testing were reviewed and referenced. Gain, single PE resolution, PDE, transit time spread (TTS) and dark count rate (DCR) are the most important parameters measured in recent different experiments. For example, Daya Bay used 8-inch dynode PMTs (9354KB, R5912, XP1806) \cite{DayaBayTesting}, Double Chooze used 10-inch dynode PMTs (R7081) \cite{DoubleChoozeTesting}, LHAASO used 8-inch dynode PMTs (CR365-02-1) \cite{LHAASOTesting}, HyperKamiokande tested 20-inch dynode PMT (R12860) and MCP-PMT (GDB-6203) \cite{HyperKTesting}, KM3Net used 3-inch dynode PMT(R12199-02) \cite{KM3NetTesting}, XENON1T and XENONnT tested 3-inch dynode PMT (R11410-21) \cite{XENON1TTesting}\cite{XENONnTTesting}, IceCube tested 3-inch PMT (R12199-01) \cite{IceCubeTesting}.

% The structure of paper
This work concentrated on the characteristics of MCP-PMTs in the low photon intensity condition. The setup of the testing system and facility is introduced in sec.~\ref{SetUp}. Detailed analysis methods and results, including Gain, single PE resolution, PDE, TTS, DCR, time characteristics of single electron response (SER), and pre-pulse and after-pulse are described in sec.~\ref{Method}. The boost for energy resolution is discussed in sec.~\ref{Result}. Finally, a summary is given in sec.~\ref{Summary}.