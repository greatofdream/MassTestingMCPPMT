\section{Experimental setup and testing procedures}
\label{SetUp}
\subsection{Experimental setup}
\label{sec:setup}
% Facility
\begin{figure}[!htbp]
    \centering
    \includegraphics[width=\SF\textwidth]{figures/facility/schema.pdf}
    \caption{The schematic figure of the testing system. The HV, laser, and FADC are controlled by the computer directly. The trigger signal from laser and waveforms from PMT are sampled by FADC. The laser light is split into 4 channels and diffused after attenuation.}
    \label{fig:facility}
\end{figure}

The schema of the experimental facility is shown in Fig.~\ref{fig:facility}. The light source is a picosecond laser flashing system (PiL040XSM) from Advanced Laser Systems (A.L.S) \cite{NTKLaser}, which can produce short light pulses with a wavelength of \SI{405}{nm} and a width of \SI{34}{ps}. Besides, a period trigger signal output from the laser is connected to the digitizer, which serves as the trigger input. An attenuator is used to adjust the laser intensity.
% The light is attenuated by a laser attenuator with an adjustable range of $[0.5,30]$\,DB.

CAEN V1751 10-bit digitizer with 1GHz sample rates is used for data taking \cite{CAENV1751}, which is controlled by a custom-made data acquisition (DAQ) software\footnote{Github repo: \url{https://github.com/greatofdream/CAENReadout}} based on CAEN library. The dynamic range of V1751 is \SI{1}{V} and \SI{1}{LSB} (Least Significant Byte) is about \SI{0.978}{mV}, which is defined as \SI{1}{ADC}. To handle the anomalous baseline larger than 0V, the offset of the digitizer is set to about \SI{0.183}{V} and the dynamic range is from \SI{-0.817}{V} to \SI{0.183}{V}. Wiener EDS 30330p high voltage (HV) module \cite{WIENERHV} supplies a positive voltage for each PMT. HV and laser are controlled by a computer using SNMP \cite{SNMP} and VISA \cite{VISA} protocol.

A black plastic box split into 4 grids with clapboards acts as a darkroom. A fiber splitter %with the standard G.657.A1 
distributes attenuated light into 4 channels. Each grid holds a PMT and is equipped with an end of splitter channel, which faces directly to the top of the PMT. The end of the splitter channel is covered with a \SI{4}{cm\tothe{2}} square diffuser plate to diffuse the laser light into a small area. Potted PMTs are connected from an integrated HV-signal decoupler in the PMT base to the HV module and digitizer with the HV lines and LEMO lines. Optical fibers and LEMO lines go through a hole with a diameter of \SI{0.5}{cm} on the box. Double-way connectors are used to connect HV lines through the box. Besides, CR365 PMT \cite{BJBS} from HAMAMATSU is used as a reference PMT with \SI{2.4}{ns} of TTS and 25\%@\SI{420}{nm} of quantum efficiency for design. %, of which PDE is larger than 22\%\cite{CR365PDE}.
This work tested 9 MCP-PMTs in the testing system.

\subsection{Testing procedures}
\begin{figure}
    \centering
    \includegraphics[width=0.4\textwidth]{figures/facility/procedure.pdf}
    \caption{The flow chart of testing procedures. PMTs are cooling down before running of DAQ. The procedures are executed by the software automaticaly.}
    \label{fig:testingprocedure}
\end{figure}

The testing procedures are indicated in Fig.~\ref{fig:testingprocedure}. All PMTs were measured under the HV provided by the vendor (NNVT). %, which has been calibrated at the gain of $1\times10^7$. 
The PMT-box map is stored as meta data which is mainly used for PDE and testing system calibration. To stabilize and reduce the influence of the dark noise, all PMTs stay in the darkroom with HV on for at least 12 hours before running of DAQ software. %Testing contains two stages: the first stage is the noise stage, in which PMTs work without laser and the digitizer is triggered periodly at 1kHz mainly for DCR calculation; the second stage is laser stage, in which 
The digitizer samples data with laser on and is triggered by the laser during the experiment. The waveforms are stored in ROOT files and analyzed with the custom-made software\footnote{Github repo: \url{https://github.com/greatofdream/pmtTest}}.
