\section{Experimental setup and test procedure}
\label{SetUp}
\subsection{Experimental setup}
\label{sec:setup}
% Facility
\begin{figure}[!htbp]
    \centering
    \includegraphics[width=0.8\textwidth]{figures/facility/schema.pdf}
    \caption{The schema of facility}
    \label{fig:facility}
\end{figure}

The schema of experimentail facility is shown in Fig.~\ref{fig:facility}. A black plastic box splitted into 4 grids acts as darkroom. Each grid holds a PMT and is equipped with an end of splitter channel. CR365-01 \cite{BJBS} and low background R5912-10WA-10 \cite{JPBS} PMTs from HAMAMATSU are used as reference PMT. Optical fibers, LEMO lines and HV lines are go through a \SI{0.5}{cm} hole of the box.

Light source is a picosecond laser flashing system (PiL040XSM) from Advanced Laser Systems(A.L.S) \cite{NTKLaser}, which can produce short light pulses with wavelength of \SI{405}{nm} and a width of \SI{34}{ps}. Besides, a period trigger signal ouput from laser is connected to digitizer, which serves as trigger when laser is on. The light is attenuated by a laser attenuator with adjustable range of 0.5-30\ db. A fiber splitter %with the standard G.657.A1 
distributes attenuated light into 4 channels, which face directly to the top of PMT.

CAEN V1751 10-bit digitizer with 1GHz sample rates is used for data taking \cite{CAENV1751}, which controlled by a custom-made data acquisition (DAQ) software\footnote{Github repo: \url{https://github.com/greatofdream/CAENReadout}} based on CAEN library. Dynamic range of V1751 is \SI{1}{V} and \SI{1}{ADC}\footnote{\SI{1}{LSB} (Least Significant Byte) is defined as \SI{1}{ADC}} equals to \SI{0.978}{mV}. To handle the anomalous baseline large than 0V, the offset of digitizer is set to about \SI{0.183}{V} and the dynamic range is from \SI{-0.817}{V} to \SI{0.183}{V}. Wiener EDS 30330p high voltage(HV) module \cite{WIENERHV} supplies a positive voltage for each PMT. Potted PMTs are connected from a HV divider with an integrated HV-signal decoupler in PMT base to the HV module and digitizer with HV line and LEMO line.

\subsection{Testing procedure}
All PMTs were measured under the HV provided by the vendor (NNVT), which has been calibrated at the gain of $1\times10^7$. To stabilize and reduce the influence of the dark noise, all pmts stayed in darkroom with HV on for at least 12 hours before running of DAQ software. Testing contains two stage: the first stage is noise stage and PMT works without laser and digitizer is triggered periodly at 1kHz mainly for DCR calculation; the second stage is laser stage, in which digitizer samples data with laser on and triggered by the output of trigger from laser. The waveforms are analyzed with custom-made software\footnote{Github repo: \url{https://github.com/greatofdream/pmtTest}}.