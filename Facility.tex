\section{Experimental setup and testing procedure}
\label{SetUp}
\subsection{Reference PMT}
CR365-01\cite{BJBS} and R5912\cite{JPBS} from HAMAMATSU are used as reference PMT.
\subsection{Experimental setup}
% Facility
The schema of experimentail facility is shown in Fig~\ref{fig:facility}
\begin{figure}
    % \includegraphics[width=0.8\textwidth]{figure/facility/facility.pdf}
    \caption{The schema of facility}
    \label{fig:facility}
\end{figure}
Light source is a picosecond laser flashing system (PiL040XSM) from Advanced Laser Systems(A.L.S)\cite{NTKLaser}, which can produce short light pulses with a wavelenght of \SI{405}{nm} and a width of \SI{34}{ps}. The light is attenuated by a laser attenuator with 0.5-30db. A fiber splitter with the standard G.657.A1 distributes attenuated light into 4 channels, which face directly to the top of PMT.

A black plastic box acts as darkroom splitted into 4 grids. Each grid holds a PMT and is equipped with an end of splitter channel. CAEN V1751 digitizer with 1GHz sample rates is used for data taking\cite{CAENV1751}, which controlled by a custom-made data acquisition(DAQ) software based on CAEN library. Wiener EDS 30330p high voltage(HV) module\cite{WIENERHV} supplies a positive voltage for each PMT. Bare PMTs are connected to the HV module and digitizer with a pluggable HV divider with an integrated HV-signal decoupler. The HV divider is soldered to potted PMTs with HV and signal interfaces seperated. 

LEMO line and HV line are go through an \SI{0.5}{cm} hole of the box and direclty connected HV divider with HV module and digitizer. 