\section{Experimental setup and test procedures}
\label{SetUp}
\subsection{Experimental setup}
\label{sec:setup}
% Facility
\begin{figure}[!htbp]
    \centering
    \includegraphics[width=\SF\textwidth]{figures/facility/schema.pdf}
    \caption{The schematic figure of the testing system. The HV module, laser, and FADC are controlled by the computer directly. The trigger signal from laser and waveforms from PMT are sampled by FADC. The laser light is split into 4 channels and diffused after attenuation.}
    \label{fig:facility}
\end{figure}

The schema of the experimental facility is shown in Fig.~\ref{fig:facility}. CAEN V1751 10-bit digitizer with 1GHz sample rates is used for data taking \cite{CAENV1751}. The dynamic range of V1751 is \SI{1}{V} and has 1024 quantisation levels, in which 1 quantisation level is denoted as \SI{1}{ADC} \cite{JUNOPrototype} in following sections. Wiener EDS 30330p high voltage (HV) module \cite{WIENERHV} supplies a positive voltage for each PMT. The light source is a picosecond laser flashing system (PiL040XSM) from Advanced Laser Systems (A.L.S) \cite{NTKLaser} and can produce short light pulses with a wavelength of \SI{405}{nm} and a width of \SI{34}{ps}. Besides, a period trigger signal output from the laser acts as the trigger input of digitizer. An attenuator is used to adjust the laser intensity. The digitizer, the HV module, and the laser are controlled by a custom-made data acquisition (DAQ) software\footnote{Github repo: \url{https://github.com/greatofdream/CAENReadout}} based on CAEN library \cite{CAENLIB}, SNMP \cite{SNMP}, and VISA \cite{VISA} protocol.

A black plastic box split into 4 grids with clapboards acts as a darkroom. A fiber splitter distributes attenuated light into 4 channels. Each grid contains a PMT and is equipped with an end of splitter channel covered with a \SI{4}{cm\tothe{2}} square diffuser plate to diffuse the laser light into a small area and faces directly to the top of the PMT. Those PMTs are connected from an integrated HV-signal decoupler in the PMT base to the HV module and digitizer with the HV lines and signal lines. Besides, CR365 PMT \cite{BJBS} from HAMAMATSU is used as a reference PMT with design specification as \SI{2.4}{ns} of TTS and 25\%@\SI{420}{nm} of quantum efficiency. This work tested 9 MCP-PMTs in the testing system.

\subsection{Test procedures}
\begin{figure}
    \centering
    \includegraphics[width=0.4\textwidth]{figures/facility/procedure.pdf}
    \caption{The flow chart of test procedures. PMTs are cooling down before running of DAQ. The procedures are executed by the software automaticaly.}
    \label{fig:testingprocedure}
\end{figure}

The test procedures are indicated in Fig.~\ref{fig:testingprocedure}. All PMTs were measured under the HV provided by the vendor (NNVT). The PMT-box map is stored as meta data which is mainly used for PDE and testing system calibration. To reduce the influence of the dark noise, all PMTs stay in the darkroom with HV on for at least 12 hours before running of DAQ software. The digitizer samples data with laser on during the experiment. The waveforms are stored in ROOT files and analyzed with the custom-made software\footnote{Github repo: \url{https://github.com/greatofdream/pmtTest}}.
