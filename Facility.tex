\section{Experimental setup and test procedure}
\label{SetUp}
\subsection{Experimental setup}
\label{sec:setup}
% Facility
\begin{figure}[!htbp]
    \centering
    \includegraphics[width=0.7\textwidth]{figures/facility/schema.pdf}
    \caption{The schema of facility. HV, laser, and FADC are controlled by computer directly. Trigger wave from laser and waveform from PMT are sampled by FADC. The laser light is splitted into 4 channel and diffused after attenuated.}
    \label{fig:facility}
\end{figure}

The schema of the experimental facility is shown in Fig.~\ref{fig:facility}. Light source is a picosecond laser flashing system (PiL040XSM) from Advanced Laser Systems(A.L.S) \cite{NTKLaser}, which can produce short light pulses with a wavelength of \SI{405}{nm} and a width of \SI{34}{ps}. Besides, a period trigger signal output from the laser is connected to the digitizer, which serves as trigger input when the laser is on. The light is attenuated by a laser attenuator with an adjustable range of 0.5-30\ DB.

CAEN V1751 10-bit digitizer with 1GHz sample rates is used for data taking \cite{CAENV1751}, which is controlled by a custom-made data acquisition (DAQ) software\footnote{Github repo: \url{https://github.com/greatofdream/CAENReadout}} based on CAEN library. The dynamic range of V1751 is \SI{1}{V} and \SI{1}{ADC}\footnote{\SI{1}{LSB} (Least Significant Byte) is defined as \SI{1}{ADC}} equals to \SI{0.978}{mV}. To handle the anomalous baseline large than 0V, the offset of digitizer is set to about \SI{0.183}{V} and the dynamic range is from \SI{-0.817}{V} to \SI{0.183}{V}. Wiener EDS 30330p high voltage(HV) module \cite{WIENERHV} supplies a positive voltage for each PMT. HV and laser are controlled by a computer using SNMP\cite{SNMP} and VISA\cite{VISA} protocol.

A black plastic box split into 4 grids acts as a darkroom. Each grid holds a PMT and is equipped with an end of splitter channel. A fiber splitter %with the standard G.657.A1 
distributes attenuated light into 4 channels, which face directly to the top of PMT. Potted PMTs are connected from an HV divider with an integrated HV-signal decoupler in the PMT base to the HV module and digitizer with the HV line and LEMO line. The end of splitter channel is covered with a \SI{4}{cm\tothe{2}} square diffuser plate to diffuse the laser light into a small area. Optical fibers, LEMO lines, and HV lines go through a hole with a diameter of \SI{0.5}{cm} on the darkroom. Besides, CR365-01 \cite{BJBS} PMT from HAMAMATSU is used as reference PMT.
\subsection{Testing procedure}
\begin{figure}
    \centering
    \includegraphics[width=0.5\textwidth]{figures/facility/procedure.pdf}
    \caption{Testing procedure flow chart. PMTs are cooling down after load in darkroom. The procedure are executed by software automaticly.}
    \label{fig:testingprocedure}
\end{figure}
The testing procedure is indicated in Fig.~\ref{fig:tesgingprocedure}. All PMTs were measured under the HV provided by the vendor (NNVT), which has been calibrated at the gain of $1\times10^7$. To stabilize and reduce the influence of the dark noise, all PMTs stay in the darkroom with HV on for at least 12 hours before running of DAQ software. Testing contains two stages: the first stage is the noise stage. PMTs work without laser and the digitizer is triggered periodly at 1kHz mainly for DCR calculation; the second stage is laser stage, in which digitizer samples data with laser on and triggered by the laser. The waveforms are analyzed with custom-made software\footnote{Github repo: \url{https://github.com/greatofdream/pmtTest}}.