\section{Experimental setup and procedures}
\label{SetUp}
The schematics of the PMT-test facility is shown in Fig.~\ref{fig:facility}. CAEN V1751 \SI{1}{GS \per s} 10-bit digitizer is used to acquire data~\cite{CAENV1751}. With the dynamic range of 1V, we use the unit of ``1 ADC''~\cite{JUNOPrototype} to represent a quantization level of 1000/1024\,mV in the following sections. Wiener EDS 30330p high voltage (HV) module~\cite{WIENERHV} supplies a positive voltage for each PMT. A picosecond laser~(PiL040XSM) from Advanced Laser Systems~\cite{NTKLaser} produces \SI{34}{ps} light pulses at \SI{405}{nm} to illuminate the PMTs via an attenuator and feeds an electronic trigger signal into the digitizer. The digitizer, the HV module and the laser are controlled by self-developed data acquisition (DAQ) software\footnote{Github repo: \url{https://github.com/greatofdream/CAENReadout}.} based on the CAENDigitizer~\cite{CAENLIB}, Net-SNMP~\cite{SNMP} and PyVISA~\cite{VISA} libraries.

% Facility
\begin{figure}[!htbp]
    \centering
    \includegraphics[width=\SF\textwidth]{figures/facility/schema.pdf}
    \caption{The schematics of the test system. The computer controls the HV module, laser, and FADC directly. The trigger signal from laser and waveforms from PMT are sampled by the FADC. The laser light is split into four channels and diffused after attenuation.}
    \label{fig:facility}
\end{figure}

PMTs are installed in dark rooms made of a black light-tight plastic box separated by extruded polystyrene boards into four parts. A fiber splitter distributes attenuated laser light into the four dark rooms. Each channel ends with a \SI{4}{cm\tothe{2}} diffuser plate to spread light onto the top of the PMT photocathode.
A base distributes HV to the pins of each PMT and outputs the amplified pulse from the anode. A CR365 PMT~\cite{BJBS} from Beijing Hamamatsu Photon Techniques Inc. is used as a reference for PDE measurements.  It has a specification of \SI{25}{\percent} quantum efficiency~(QE) at \SI{420}{nm}, and \(0.97 \times \SI{25}{\percent}\) at \SI{405}{nm}~\cite{HAMAMATSUManual}. The PDE at \SI{405}{nm} is estimated to be 17\%, corresponding to a collection efficiency of 0.7--0.8 typical for 8-inch dynode PMTs~\cite{WANG2012113,R5912MOD,RCESpotlight}.

\begin{figure}
    \centering
    \includegraphics[width=0.4\textwidth]{figures/facility/procedure.pdf}
    \caption{The flow chart of test procedures. PMTs are cooled down before DAQ. The procedures are executed automatically.}
    \label{fig:testingprocedure}
\end{figure}

The test procedures indicated in Fig.~\ref{fig:testingprocedure} are executed by DAQ software automatically. To lower the systematic errors from the light source variation and unknown splitter ratios, we permute the PMTs in the dark rooms to conduct PDE measurements and light source calibration simultaneously in Section~\ref{sec:PDE}. To cool down the DCR, all the PMTs stay in the darkroom with vendor-specified HV for at least 12 hours before the digitizer acquires waveforms with laser on. The waveforms are stored in ROOT~\cite{brun_root_1997} files and analyzed with self-developed software\footnote{Github repo: \url{https://github.com/greatofdream/pmtTest}.}.
