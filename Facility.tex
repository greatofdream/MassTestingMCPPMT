\section{Experimental setup and test procedures}
\label{SetUp}
\subsection{Experimental setup}
\label{sec:setup}
% Facility
\begin{figure}[!htbp]
    \centering
    \includegraphics[width=\SF\textwidth]{figures/facility/schema.pdf}
    \caption{The schematics of the test system. The HV module, laser, and FADC are controlled by the computer directly. The trigger signal from laser and waveforms from PMT are sampled by the FADC. The laser light is split into 4 channels and diffused after attenuation.}
    \label{fig:facility}
\end{figure}

The schema of the experimental facility is shown in Fig.~\ref{fig:facility}. CAEN V1751 1\,GS/s 10-bit digitizer sample rates is used for data taking \cite{CAENV1751}. With the dynamic range of 1V, we use the unit of ``1 ADC''  to represent a quantization level of 1000/1024\,mV \cite{JUNOPrototype} in following sections. Wiener EDS 30330p high voltage (HV) module \cite{WIENERHV} supplies a positive voltage for each PMT. The light source is a picosecond laser (PiL040XSM) from Advanced Laser Systems (A.L.S) \cite{NTKLaser} producing light pulses of \SI{34}{ps} at \SI{405}{nm} wavelength. Besides, a period trigger signal output from the laser is fed into digitizer. An attenuator is used to adjust the laser intensity. The digitizer, the HV module, and the laser are controlled by self-developed data acquisition (DAQ) software\footnote{Github repo: \url{https://github.com/greatofdream/CAENReadout}} based on CAEN library \cite{CAENLIB}, SNMP \cite{SNMP}, and VISA \cite{VISA} protocol.

PMTs are installed into dark rooms made of a black light-tight plastic box seperated by extruded polystyrene boards into 4 parts. A fiber splitter distributes attenuated laser light into the 4 dark rooms. Each channel ends with a \SI{4}{cm\tothe{2}} diffuser plate to spread light onto the top of PMT photocathode.
A base distributes HV to the pins of each PMT and outputs the amplified pulse from the anode. A CR365 PMT \cite{BJBS} from HAMAMATSU is used as a reference with design specification as \SI{2.4}{ns} of TTS and 25\%@\SI{420}{nm} of quantum efficiency.

\subsection{Test procedures}
\begin{figure}
    \centering
    \includegraphics[width=0.4\textwidth]{figures/facility/procedure.pdf}
    \caption{The flow chart of test procedures. PMTs are cooled down before DAQ. The procedures are executed automatically.}
    \label{fig:testingprocedure}
\end{figure}

The test procedures indicated in Fig.~\ref{fig:testingprocedure} are executed by DAQ software automatically. To lower the systematics by the light source variation and unknown splitter ratios, we circulate the PMTs in the dark rooms and store the PMT-channel map as metadata to do PDE measurements and light source calibration in Section~\ref{sec:PDE}. To cool down the DCR, all the PMTs stay in the darkroom with vendor specified HV for at least 12 hours before the digitizer acquiring waveforms with laser on. The waveforms are stored in ROOT files and analyzed with the self-developed software\footnote{Github repo: \url{https://github.com/greatofdream/pmtTest}}.
