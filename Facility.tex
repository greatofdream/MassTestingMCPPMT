\section{Experimental setup and testing procedure}
\label{SetUp}
\subsection{Experimental setup}
\label{sec:setup}
% Facility
CR365-01\cite{BJBS} and low background R5912-10WA-10\cite{JPBS} PMTs from HAMAMATSU are used as reference PMT.

The schema of experimentail facility is shown in Fig~\ref{fig:facility}

\begin{figure}[!htbp]
    \centering
    \includegraphics[width=0.8\textwidth]{figures/facility/schema.pdf}
    \caption{The schema of facility}
    \label{fig:facility}
\end{figure}

Light source is a picosecond laser flashing system (PiL040XSM) from Advanced Laser Systems(A.L.S)\cite{NTKLaser}, which can produce short light pulses with a wavelength of \SI{405}{nm} and a width of \SI{34}{ps}. The light is attenuated by a laser attenuator with 0.5-30db. A fiber splitter with the standard G.657.A1 distributes attenuated light into 4 channels, which face directly to the top of PMT.

A black plastic box acts as darkroom splitted into 4 grids. Each grid holds a PMT and is equipped with an end of splitter channel. Optical fiber, LEMO line and HV line are go through an \SI{0.5}{cm} hole of the box.

CAEN V1751 10-bit digitizer with 1GHz sample rates is used for data taking\cite{CAENV1751}, which controlled by a custom-made data acquisition(DAQ) software based on CAEN library. Dynamic range of V1751 is \SI{1}{V} and \SI{1}{ADC}\footnote{\SI{1}{LSB} (Least Significant Byte) is defined as \SI{1}{ADC}} equals to \SI{0.978}{mV}. To handle the anomalous baseline large than 0V, the offset of digitizer is set to about \SI{0.183}{V} and the dynamic range is from \SI{-0.817}{V} to \SI{0.183}{V}. Wiener EDS 30330p high voltage(HV) module\cite{WIENERHV} supplies a positive voltage for each PMT. Potted PMTs are connected from a HV divider with an integrated HV-signal decoupler in PMT base to the HV module and digitizer with HV line and LEMO line.


\subsection{Testing procedure}
All PMTs were measured under the HV provided by the vendor(NNVT), which are calibrated at the gain of $1\times10^7$. To stabilize and reduce the influence of the dark noise, all pmts stayed in darkroom  with HV on for 10 hours before DAQ started. DAQ sampling contains two stage: the first stage is noise stage and DAQ works without laser and triggered periodly at 1kHz mainly for DCR calculation; the second stage is laser stage, in which DAQ sample data with laser and triggered by the output of trigger from laser for other characterization.