\begin{abstract}
    Jinping Neutrino Experiment (JNE) plans to deploy a new type of 8-inch MCP-PMT with high photon detection efficiency for MeV-scale neutrino measurements. This work presents the test procedures for the characteristics of the MCP-PMTs, compared to a commercial 8-inch dynode PMT. Tests demonstrate that the photon detection efficiency, the transition time spread, and the charge peak-valley ratio of the new type MCP-PMTs are significantly better. We find a long tail in the charge distribution of single PE, but combined with the high photon detection efficiency, the overall energy resolution sees substantial improvements. Single photoelectron response, rates of dark counts and after pulses are also provided as the inputs to detector simulation and design. Our results show that this PMT satisfies all the requirements of JNE.

\end{abstract}

\begin{keyword}
    MCP-PMT \sep photon detection efficiency \sep Jinping Neutrino Experiment
    %% PACS codes here, in the form: \PACS code \sep code
    \PACS 29.40.Mc
    %% MSC codes here, in the form: \MSC code \sep code
    %% or \MSC[2008] code \sep code (2000 is the default)
\end{keyword}
