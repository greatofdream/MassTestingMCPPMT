\begin{abstract}
    Jinping Neutrino Experiment (JNE) plans to use a new type of 8-inch MCP-PMT with high photon detection efficiency for MeV-scaled neutrino measurements. This work presents the implemented testing procedures and the characteristics of a set of MCP-PMTs, compared to a commercial 8-inch dynode PMT with typical quantum efficiency. Tests demonstrate that the photon detection efficiency, the transition time spread, and the charge peak-valley ratio are significantly better than the reference PMT. Considering the influence of long tail in charge distribution, the boost of photon detection efficiency still contributes to the detector energy resolution in theory. All measured parameters can be used in the numerical simulation for further research. The new type of MCP-PMT is promising in the MeV-scaled neutrino experiments. 
\end{abstract}

\begin{keyword}
    MCP-PMT \sep photon detection efficiency \sep Jinping Neutrino Experiment
    %% PACS codes here, in the form: \PACS code \sep code
    \PACS 29.40.Mc
    %% MSC codes here, in the form: \MSC code \sep code
    %% or \MSC[2008] code \sep code (2000 is the default)
\end{keyword}
